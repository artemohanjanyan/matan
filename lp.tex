\section{Пространства $L_p$}

\begin{definition}
	$L_p(E) \defeq \left\{f : E \rightarrow \mathbb{R}, f - \text{измерима}, \intl{E} \abs{f}^p < +\infty \right\}$
\end{definition}

Нам нужно проверить, что $L_p(E)$ - НП. То есть, $f,g \in L_p \Rightarrow \alpha f + \beta g \in L_p$, $\norm{f}_p = \left( \intl{E} \abs{f}^p \right)^{\frac{1}{p}}$.
Причем, $\norm{f}$ - удовлетворяет аксиомам нормы. 

\begin{statement}
	$\normp{f}$ удовлетворяет двум свойствам:
	\begin{enumerate}
		\item
			$\normp{\alpha f } = |\alpha| \normp{f}$
		\item
			$\normp{f + g} \leqslant \normp{f} + \normp{g}$
	\end{enumerate}
\end{statement}

\begin{proof}
	\begin{enumerate}
		\item
			Очевидно.
		\item
			$\intl{E} \abs{f + g}^p \leqslant \intl{E} (\abs{f} + \abs{g})^p$. Пусть $E_1 = E(\abs{f} \leqslant \abs{g}), \: E_2 = E(\abs{f} > \abs{g})$.
			Тогда $E = E_1 \cup E_2$. 
			\begin{gather*}
				\intl{E} (\abs{f} + \abs{g})^p = \intl{E_1} (\abs{f} + \abs{g})^p + \intl{E_2} (\abs{f} + \abs{g})^p \leqslant \\
				\leqslant \intl{E_1} (2\abs{g})^p + \intl{E_2} (2\abs{f})^p < +\infty
			\end{gather*}
			Следовательно $f + g \in L_p$
	\end{enumerate}
\end{proof}

\subsection{Неравенство Гельдера}

\begin{theorem}[Неравенство Гельдера]
	Пусть $p > 1$ и $q : \frac{1}{p} + \frac{1}{q} = 1$. Пусть $f \in L_p, \: g \in L_q$. Тогда
	\begin{gather*}
		\intl{E} \abs{f}\abs{g} \leqslant \left( \intl{E} \abs{f}^p \right)^{\frac{1}{p}}  \left( \intl{E} \abs{g}^q \right)^{\frac{1}{q}}
	\end{gather*}
\end{theorem}


\begin{proof}
	Воспользуемся неравенством Юнга: ($uv \leqslant \frac{1}{p}u^p + \frac{1}{q}v^q$).
	Пусть $u =  \frac{\abs{f}}{\normp{f}}$, $v = \frac{\abs{g}}{\normp{g}}$
	\begin{gather*}
		\frac{\abs{f}\abs{g}}{\normp{f}\normp{g}} \leqslant \frac{1}{p} \frac{\abs{f}^p}{\normp{f}^p} + \frac{1}{q} \frac{\abs{g}^q}{\normp{g}^q}\\
		\intl{E}\frac{\abs{f}\abs{g}}{\normp{f}\normp{g}} \leqslant 
		\frac{1}{p} \intl{E}\frac{\abs{f}^p}{\normp{f}^p} + 
		\frac{1}{q} \intl{E}\frac{\abs{g}^q}{\normp{g}^q} = \frac{1}{p} + \frac{1}{q} = 1 
	\end{gather*}
\end{proof}

\subsection{Неравенство Минковского}

\begin{theorem}[Неравенство Минковского]
	Пусть $p > 1$, $f, g \in L_p$. Тогда
	\begin{gather*}
		\left(\intl{E} (\abs{f} + \abs{g})^p \right)^{\frac{1}{p}} \leqslant 
		\left( \intl{E} \abs{f}^p \right)^{\frac{1}{p}} + 
		\left( \intl{E} \abs{g}^p \right)^{\frac{1}{p}}
	\end{gather*}
\end{theorem}

\begin{proof}
	Рассмотрим $(f + g)^p = f(f + g)^{p - 1} + g(f + g)^{p - 1}$.
	\begin{gather*}
		\intl{E} (f + g)^p = \intl{E} f(f + g)^{p - 1} + \intl{E}g(f + g)^{p - 1} \leqslant \\
		\leqslant \left( \intl{E} f^p \right)^{\frac{1}{p}} \left(\intl{E} (f + g)^{q(p - 1)} \right)^{\frac{1}{q}} + 
	  		  \left( \intl{E} g^p \right)^{\frac{1}{p}} \left(\intl{E} (f + g)^{q(p - 1)} \right)^{\frac{1}{q}} \\
		\text{пусть}\: q = \frac{p}{p - 1} \\
		\left(\intl{E}(f + g)^p \right)^{1 - \frac{1}{q}} \leqslant  \left( \intl{E} f^p \right)^{\frac{1}{p}} +  \left( \intl{E} g^p \right)^{\frac{1}{p}} \\
		\frac{1}{p} = 1 - \frac{1}{q}
	\end{gather*}
\end{proof}

Если подставить в неравенство Минковского определение нормы, то можно заметить, что мы доказали неравенство треугольника.

\begin{theorem}
	$L_p(E) $ - Банахово пространство.
\end{theorem}

Докажем вспомогательную лемму:

\begin{lemma}
	Пусть $f_n$ - измеримы, и $\forall\delta > 0\: \mu E\left(\abs{f_n - f_m} \geqslant \delta\right) \xrightarrow[n,m \rightarrow +\infty]{} 0$.
	Тогда $\exists n_1 < n_2 < \dots < n_k < \dots : f_{n_k} \rightarrow f$ почти всюду.
\end{lemma}

\begin{proof}
	Пусть $\delta = \frac{1}{2^k}$. Можно проверить, что (\todo) 
	$\exists n_1 < n_2 < \dots < n_k < \dots: \\$
	$\mu E\left(\abs{f_{n_{k + 1}} - f_{n_k}} \geqslant \frac{1}{2^k}\right) \leqslant \frac{1}{2^k}$.
	Рассмотрим следующее множество:
	\begin{gather*}
		E' = \bigcup\limits_{k = 1}^{\infty}\bigcap\limits_{j = k}^{\infty} E\left(\abs{f_{n_{j + 1}} - f_{n_j}} \leqslant \frac{1}{2^j} \right)
	\end{gather*}
	Рассмотрим функциональный ряд $S = f_1 + (f_2 - f_1) + (f_3 - f_2) + \dots$. Фиксируем $x \in E'$. Тогда 
	$\\ \exists k_x : x \in \bigcap\limits_{j = k_x}^{\infty} E\left(\abs{f_{n_{j + 1}} - f_{n_j}} \leqslant \frac{1}{2^j} \right)$.
	Это значит, что при $j > k_x$ выполняется $\abs{f_{n_{j + 1}}(x) - f_{n_k}(x)} \leqslant \frac{1}{2^j} \xrightarrow[j \rightarrow +\infty]{} 0$.
	Следовательно, на $E'$ функциональный ряд $S$ - сходится. 
	Нам осталось проверить, что его дополнение нуль-мерно.
	Т. е. $\mu \overline{E'} = 0$. Очевидно:
	\begin{gather*}
		\overline{E'} = \bigcap\limits_{k = 1}^{\infty}\bigcup\limits_{j = k}^{\infty} E\left(\abs{f_{n_{j + 1}} - f_{n_j}} > \frac{1}{2^j} \right) \Rightarrow \\
		\Rightarrow \overline{E'} \subset \bigcup\limits_{j = k}^{\infty} E\left(\abs{f_{n_{j + 1}} - f_{n_j}} > \frac{1}{2^j} \right) \Rightarrow \\
		\Rightarrow \mu \overline{E'} \leqslant \sumlr{j = k}{\infty} \mu E(\abs{f_{n_{j + 1}} - f_{n_j}} > \frac{1}{2^j}) \leqslant 
		\sumlr{j = k}{\infty} \frac{1}{2^j} \xrightarrow[k \rightarrow +\infty]{} 0 \Rightarrow \\
		\Rightarrow \mu \overline{E'} = 0
	\end{gather*}
\end{proof}

\newpage

\begin{proof}[Доказательство Теоремы]
	Докажем для случая $p = 1$ (общий случай напишу потом \todo).
	Итак, $f_n \in L_1(E), \normpp{f_n - f_m}{1} \xrightarrow[n,m \rightarrow +\infty]{} 0$. Зафиксируем $\forall\delta > 0$. 
	Тогда 
	\begin{gather*}
		\delta\mu E(\abs{f_n - f_m} \geqslant \delta) \leqslant 
		\intl{E(\abs{f_n - f_m} \geqslant \delta)} \abs{f_n - f_m} \leqslant \intl{E} \abs{f_n - f_m} = \normpp{f_n - f_m}{1}
		\rightarrow 0
	\end{gather*}
	Отсюда, по лемме, $\exists(n_1 < n_2 < \dots < n_k < \dots) : f_{n_k} \xrightarrow[k \rightarrow +\infty]{} f$ почти всюду на $E$.
	Коль скоро $k$ - фиксированное, $\abs{f_{n_k} - f_{n_m}} \xrightarrow[m \rightarrow +\infty]{} \abs{f_{n_k} - f}$ почти всюду на $E$.
	По теореме Фату:
	\begin{gather*}
		\intl{E}\abs{f_{n_k} - f} \leqslant \sup\limits_{m}\intl{E} \abs{f_{n_k} - f_{n_m}}
	\end{gather*}
	В силу $\normpp{f_{n_k} - f_{n_m}}{1}\xrightarrow[k,m \rightarrow +\infty]{} 0$
	\begin{gather*}
		\feps>0\exists M : \forall k, m > M \Rightarrow \normpp{f_{n_k} - f_{n_m}}{1}\leqslant \Epsilon
	\end{gather*}
	Без ограничения общности можем считать, что $k$ и $m$ удовлетворяют вышеприведенному условию.
	Получаем, что 
	\begin{gather*}
		\forall k > M \Rightarrow \intl{E} \abs{f_{n_k} - f} \leqslant \Epsilon
	\end{gather*}
	Отсюда, $f_{n_k} - f$ суммируема, а значит $\in L_1$. Но, по условию, $f_{n_k} \in L_1 \Rightarrow f \in L_1$. 
	Так же, мы знаем, что $\normpp{f_{n_k} - f}{1} \leqslant \Epsilon$, что, в свою очередь означает, что $f_{n_k} \rightarrow f$ по норме в $L_1$.
	Оценим $\normpp{f_n - f}{1}$:
	\begin{gather*}
		\normpp{f_n - f}{1} \leqslant \normpp{f_{n_k} - f_n}{1} + \normpp{f_{n_k} - f}{1} \\
		\normpp{f_{n_k} - f_n}{1} \xrightarrow[n,k \rightarrow +\infty]{} 0 \\
		\normpp{f_{n_k} - f}{1}   \xrightarrow[k \rightarrow +\infty]{} 0
	\end{gather*}
	Получаем сходимость $f_n$ к $f$ по норме в $L_1$. А значит - полноту.
\end{proof}

Может показаться, что требование измеримости функции $f$ в определении пространства $L_p$ -  излишне. Это отнюдь не так.
\begin{statement}
	Существует функция $f$ такая, что ее $p$-я степень измерима, но сама функция - нет. 
\end{statement}

\begin{proof}
	Рассмотрим произвольное неизмеримое множество $C \subset \mathbb{R}$. Тогда пусть
	\begin{gather*}
		f(x) = 
		\begin{cases}
			1  &, x \in C \\
			-1 &, x \notin C
		\end{cases}
	\end{gather*}
	Очевидно, $f$ -неизмерима (так как множество Лебега $E(f > \frac{1}{2}) = С$ - неизмеримо). Но $f^2(x) = 1$ при $x \in \mathbb{R}$ - очевидно, измеримая функция.
\end{proof}


Рaссмотрим $f, g \in L_2$. По неравенству Гельдера, их произведение суммируемо. Положим $\scalarp{f}{g} = \intl{E} f\cdot g$. 
Очевидно, таким образом построенное отображение удовлетворяет аксиомам скалярного произведения. 
Получается, что $L_2$ - гильбертово пространство с нормой, естественным образом порожденной скалярным произведением: $\normpp{f}{2} = \sqrt{\scalarp{f}{f}}$.\newline

Приведем важный частный случай пространства $L_2(E)$: В качестве тройки множество - $\sigma$-алгебра - мера возьмем: 
$\left(X, \mathscr{A}, \mu \right) = \left(\mathbb{N}, 2^{\mathbb{N}}, \mu\right)$, где $\mu$ - считающая мера (количество элементов в множестве).
Тогда $\intl{E} f = \sum\limits_{n = 1}^{\infty} f(n)$. В данном контексте суммируемость будет значить абсолютную сходимость.
\newline 
Рассмотрим $L_2(\mathbb{N})$, Обозначаем $a_n = f(n)$. Тогда $f \in L_2 (\mathbb{N}) \Leftrightarrow \sum\limits_{n} a^2_n < +\infty$. Принято бозначать $L_2(\mathbb{N}) = l_2$
\newpage


