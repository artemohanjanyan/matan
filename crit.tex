\section{Критерий Лебега интегрируемости по Риману}

\begin{definition}[Колебание на отрезке]
	\[ \omega(f, c, d) = \sup\limits_{[c, d]} f - \inf\limits_{[c, d]} f =  
    \text{(по лемме из 1го семестра)}  = \sup\limits_{x',x'' \in [c,d]} | f(x') - f(x'') |\]
\end{definition}

\begin{definition}[Колебание функции в точке]
	\[ \omega(f, x) = \lim\limits_{\delta \rightarrow 0} \omega(f, x + \delta, x - \delta)\]
\end{definition}

Очевидно, колебание на отрезке неотрицательно, и, если $0 < \delta_1 < \delta_2$ 
то $\omega(f, x - \delta_1, x + \delta_1) < \omega(f, x - \delta_2, x + \delta_2)$.
Поэтому, вышеприведенный предел существует.

\begin{statement}
    $\omega(f, x) = 0 \Leftrightarrow f \in C(x)$
\end{statement}

\begin{proof}
	\begin{enumerate}
		\item $\Leftarrow$ Раз функция непрерывна, значит она достигает на отрезке своего $\sup$ и $\inf$. 
            Значит, если устремить границы отрезка к одной точке, в пределе получим разность двух одинаковых чисел.
		\item $\Rightarrow$ $\omega(f, x) = 0$ означает, что можно подобрать такую $\delta-$окрестность для $x$, 
            что она будет сколь угодно малой. Берем формулу $\sup\limits_{x',x'' \in [x - \delta,x + \delta]} | f(x') - f(x'') | = 0$ 
            фиксируем $x'' = x$ (от этого $\sup$ разве что уменьшится) и получаем определение непрерывности в $x$.
	\end{enumerate}
\end{proof}

\begin{definition}
$\tau:$ - разбиение отрезка $[a, b]$, если $\tau = \{x_j\}: \: a = x_0 < x_1 < \dots < x_n = b$
\end{definition}

Ведем кусочно-постоянную функцию $g(\tau, x) = \omega(f, x_j, x_{j + 1}),$ при $x \in [x_j, x_{j + 1}]$

\begin{statement}
$g(\tau_n, x) \xrightarrow[n \rightarrow +\infty]{} \omega(f, x)$ почти всюду на отрезке
\end{statement}

\begin{proof}
    Очевидно, мы можем подбирать $\tau_n$ так, чтобы границы отрезка, содержащего $x$ 
    совпали с границами из определения $\omega(f,x)$. Тогда для неграничных точек получим стремление. 
    Граничных точек на конечном шаге - конечное число, а это значит, что мы не перейдем за границу счетной 
    мощности (danger zone - МАТЛОГИКА), и предел будет почти всюду 
\end{proof}

Тогда, по теореме Лебега о предельном переходе под знаком интеграла, получаем:

\begin{gather*}
    \int\limits_{[a, b]}g(\tau_n, x)dx \rightarrow \int\limits_{[a,b]}\omega(f,x)dx
\end{gather*}

Левая часть, по лемме из первого семестра равна $\int\limits_{[a, b]}g(\tau_n, x)dx = \omega(f, \tau_n)$.
Получаем:

\begin{gather*}
    \lim\limits_{rang\tau_n \rightarrow 0} \omega(f, \tau_n) = \int\limits_{[a,b]} \omega(f,x)dx
\end{gather*}

Это наша рабочая формула.

\begin{theorem}[Критерий Лебега интегрируемости по Риману]
    $\\ f \in \Re (a, b) \Leftrightarrow \lambda\{ a : f \notin C(a) \} = 0$
\end{theorem}

\begin{proof}
	\begin{enumerate}
		\item 
			$\Rightarrow \\$ Пусть $\omega(f,x) = 0$ почти всюду на $[a, b]$. Тогда $\int\limits_{[a,b]} \omega(f, x)dx = 0 \: \Rightarrow f \in \Re [a,b]$ 
		\item 
			$\Leftarrow \\$ Пусть $f \in \Re [a, b]$. Тогда, по определению, $\omega(f, \tau_n) \rightarrow 0$. 
            Тогда $\int\limits_{[a,b]} \omega(f, x)dx = 0$. Но $\omega(f, x) \geqslant 0$. 
            Значит $\omega(f,x) = 0$ почти всюду на $[a,b]$ (И, по лемме, почти всюду непрерывна).
	\end{enumerate}
\end{proof}


