\section{Предельный переход в классе суммируемых функций}

\subsection{Теорема Лебега о мажорируемой сходимости}

\begin{theorem}[Теорема Лебега о мажорируемой сходимости]
	Пусть $f_n \Rightarrow f$ на $E$, $|f_n| \leqslant \phi$ на $E$, $\phi$ - суммируема.
	$\\$ Тогда:
	\begin{enumerate}
		\item
			$f$ - суммируема
		\item
			$\intl{E} f_n \rightarrow \intl{E} f$
	\end{enumerate}
\end{theorem}

\begin{center}
	\framebox[4in]{
		\begin{minipage}[t]{3.5in}% я дезигнер
			Следует иметь ввиду, что в условии теоремы достаточно требовать выполнения свойств почти всюду.
		\end{minipage}
	}	
\end{center}

\begin{theorem}
	Пусть $f$ - суммируема на $E$. Тогда $\feps > 0 \: \exists \delta > 0 : \forall E' \subset E \Rightarrow \mu E' < \delta \Rightarrow |\intl{E'} f| < \Epsilon$ 
\end{theorem}

\begin{proof}
	По определению, можно написать $\feps > 0 \: \exists e : \intl{E \setminus e} |f| < \Epsilon$. Так как $e$ - допустимо, $f$ - ограничена на $e$ и $E = (E \setminus e) \cup e$. Возьмем любое $E' \subset E$, тогда $E' = E'(E \setminus e) \cup E'e$.
	\[
		\abs{\intl{E'} f} \leqslant \abs{\intl{E'(E \setminus e)} f} + \abs{\intl{E'e} f} \leqslant \Epsilon + \abs{\intl{E'e} f}
	\]
	Мы считаем, что $\abs{f(x)} \leqslant M$. Заметим, что выбор $E'$ не накладывал никаких ограничений на $M$. Тогда:
	\[
		\intl{E'e} \abs{f} \leqslant M \mu E'e \leqslant M \mu E'
	\]
	Поэтому $\delta$ мы можем выбрать как $\delta = \frac{\Epsilon}{M}$. И получится, что $\mu E' \leqslant \delta \Rightarrow \abs{\intl{E'} f} \leqslant 2\Epsilon$
\end{proof}

\begin{proof}[Доказательство теоремы Лебега]
	По теореме Рисса $f_{n_k} \rightarrow f$ почти всюду, причем $\abs{f_{n_k}(x)} \leqslant \phi(x)$, занчит $\abs{f(x)} \leqslant \phi(x) \: \Rightarrow f -$ суммируема. 	Рассмотрим $\abs{\intl{E} f_n  - \intl{E} f} \leqslant \intl{E} \abs{f_n - f}$. Так как $\phi $ - суммируема, $\\* \feps > 0 \:\: \exists e \text{(допустимое для $\phi$)} : \intl{E \setminus e} \phi \leqslant \Epsilon$ 
	\[
		\intl{E} \abs{f_n - f} = \intl{E \setminus e} \abs{f_n - f} + \intl{e} \abs{f_n - f} \leqslant 2\Epsilon + \intl{e} \abs{f_n - f}
	\]
	Пусть $\abs{\phi} \leqslant M \Rightarrow \abs{f_n - f} \leqslant 2M$. Так же мы знаем, что $\intl{e} \abs{f_n - f} \xrightarrow[n \rightarrow +\infty]{} 0 (*)$. Значит, начиная с некоторого $N_0$, $\intl{e} \abs{f_n - f} < \Epsilon$.
	Следовательно, начиная с $N_0$, $\intl{E} \abs{f_n - f} \leqslant 3\Epsilon$
\end{proof}
\begin{proof}[Доказательство звездочки]
	Распишем $e$:
	\begin{gather*}
		e = e(\abs{f_n - f} \geqslant \xi) \cup e(\abs{f_n - f} < \xi) = e_1 \cup e_2\\
		\Rightarrow \intl{e}\abs{f_n - f} = \intl{e_1}\abs{f_n - f} + \intl{e_2}\abs{f_n - f} \leqslant 2M \lambda e_1 + \xi \lambda e_2
	\end{gather*}
	В силу сходимости по мере, $\lambda e_1 \rightarrow 0, \: \forall \xi$, так как $\xi$ - любое, можем устремить его к нулю.
\end{proof}

\newpage

\subsection{Теорема Леви}

\begin{theorem}[Теорема Леви]
	Пусть $f_n(x) \leqslant 0$, $f_n(x) \leqslant f_{n + 1}(x)$, $f(x) = \lim\limits_{n\rightarrow +\infty} f_n(x)$ на $E$. 
	Тогда $\intl{E} f_n \rightarrow \intl{E} f$
\end{theorem}


\begin{proof}
	Два случая:
	\begin{enumerate}
		\item
			$f$ - почти всюду конечна на $E$.
			Два подслучая:
			\begin{enumerate}
				\item
					$\intl{E} f < +\infty$. Так как $\abs{f_n(x)} \leqslant f(x) \: \Rightarrow$ $f$ - суммируемая мажоранта для $f_n$, и теорема верна по теореме Лебега о мажорируемой сходимости.
				\item
					$\intl{E} f = +\infty$. ($f$ все еще мажоранта для $f_n$, но уже не суммируемая) Мы поступим так. Раз $\sup\limits_{e - допустимо} \intl{e} f = +\infty$, значит $\forall c > 0 \: \exists e - $допустимое для $f : c < \intl{e} f$. В силу $f_n \leqslant f$  по теореме Лебега о мажорируемой сходимости $\intl{e} f_n \rightarrow \intl{e} f$. Это значит, начиная с некоторого $N_0$, $c < \intl{e} f_n \leqslant \intl{E} f_n \: \Rightarrow \intl{E} f_n \rightarrow +\infty = \intl{E} f$
			\end{enumerate}
		\item
			$\mu E(f = +\infty) > 0$ (Расслабьтесь, и будет не больно)$\\$ 
			Очевидно, в этой ситуации может быть только $\intl{E} f = +\infty$. Из $f_n(x) \leqslant f_{n + 1}(x) \: 
			\Rightarrow \intl{E} f_{n}(x) \leqslant \intl{E} f_{n + 1}(x)$. 
			По теореме Вейерштрасса, у последовательности $\left\{ \intl{E} f_n \right\}$ будет существовать предел. 
			Причем он будет конечным тогда и только тогда, когда эта последовательность ограничена. 
			Так что нам нужно вывести противоречие из того факта, что эта последовательность ограничена.
			Предположим, что это так: пусть $\intl{E} f_n \leqslant M$. Итак, зафиксируем $\forall c > 0$. Рассмотрим $E(f_n \geqslant c) \subset E$.
			\begin{gather*}
				\intl{E(f_n \geqslant c)} f_n \leqslant M \\ c \mu E(f_n \geqslant c) \leqslant \intl{E(f_n \geqslant c)} f_n \Rightarrow 
				\mu E(f_n \geqslant c) \leqslant \frac{M}{c}
			\end{gather*}
			Можно проверить, что:
			\begin{gather*}
				E(f = +\infty) \subset \bigcup\limits_{m = 1}^{\infty} \bigcap\limits_{n = m}^{\infty} E(f_n \geqslant c)
			\end{gather*}
			\begin{proof}
				Пусть $x \in E(f = +\infty)$. Значит $f_n(x) \xrightarrow[n \rightarrow +\infty]{} +\infty$. 
				Следовательно $\forall c > 0 \exists N_x : \forall n > N_x \Rightarrow f_n(x) \geqslant c \xRightarrow[def]{by} 
				x \in \bigcap\limits_{n = N_x}^{\infty} E(f_n \geqslant c)$
			\end{proof}
			Заметим одну интересную штуку. 
			\begin{gather*}
				\forall c > 0 f_n(x) \geqslant c \Rightarrow f_{n + 1}(x) \geqslant c \Rightarrow 
				E(f_n \geqslant c) \subset E(f_{n + 1} \geqslant c) \Rightarrow \bigcap\limits_{n = m}^{\infty} E(f_n \geqslant c) = E(f_m \geqslant c)
				\Rightarrow \\
				\Rightarrow \bigcup\limits_{m = 1}^{\infty} \bigcap\limits_{n = m}^{\infty} E(f_n \geqslant c) = 
				\lim\limits_{m \rightarrow +\infty} E(f_m \geqslant c)
			\end{gather*}
			Отсюда делаем вывод, что: 
			\begin{gather*}
				\mu \bigcap\limits_{n = m}^{\infty} E(f_n \geqslant c) \xrightarrow[m \rightarrow +\infty]{} 
				\mu \bigcup\limits_{m = 1}^{\infty} \bigcap\limits_{n = m}^{\infty} E(f_n \geqslant c) \geqslant
				\mu E(f = +\infty)
			\end{gather*}
			\newpage
			\begin{gather*}
				\mu \bigcap\limits_{n = m}^{\infty} E(f_n \geqslant c) = \mu E(f_m \geqslant c) \leqslant \frac{M}{c} \Rightarrow \\
				\Rightarrow \mu \bigcup\limits_{m = 1}^{\infty} \bigcap\limits_{n = m}^{\infty} E(f_n \geqslant c) \leqslant \frac{M}{c} \Rightarrow \\
				\Rightarrow \mu E(f = +\infty) \leqslant \frac{M}{c}
			\end{gather*}
			$c$ - любое, поэтому можно устремить $c \rightarrow +\infty$. Значит $\mu E(f = +\infty) = 0$. Противоречие получено.
	\end{enumerate}
\end{proof}

\begin{corollary}
	Пусть $u_n(x) \geqslant 0$ и $\suml{n}\intl{E}u_n$ - сходится. Тогда $\suml{n}u_n(x)$ - сходится почти всюду на $E$.
\end{corollary}

\begin{proof}
	$S_n = u_1 + u_2 + \dots + u_n$. Так как интеграл сходится, его частичная сумма ограничена. $M \geqslant \sumlr{n = 1}{m}\intl{E} u_n = \intl{E} S_m$		
	Обозначим $S(x) = \suml{n} u_n(x)$. В силу неотрицательности $u_n(x)$, $S_n(x)$ - возрастает ($S_n \leqslant S_{n + 1}$), и 
	$S(x) = \lim\limits_{n \rightarrow +\infty} S_n(x)$. Следовательно, по теореме Леви $\intl{E} S_n \rightarrow \intl{E} S$. Следовательно, $S$ - суммируемая функция, это значит, что она почти всюду конечна. 
\end{proof}

\begin{corollary}
	Пусть $f \geqslant 0,  \:\: f_n(x) = \min\left\{f(x), n\right\}$ - срезка функции $f$. Тогда $\intl{E} f_n \rightarrow \intl{E} f$.
\end{corollary}

\begin{proof}
	$f_n$ удовлетворяют условиям теоремы Леви.
\end{proof}

\subsection{Теорема Фату}

\begin{theorem}[Теорема Фату]
	Пусть $f_n \geqslant 0$, $f_n \Rightarrow f$ на $E$. Тогда \[\intl{E} f \leqslant \sup\limits_{n \in \mathbb{N}} \intl{E} f_n	\]
\end{theorem}

\begin{proof}
	Применим теорему Рисса, получив, что $f_{n_k} \rightarrow f$ почти всюду. Без ограничения общности можем считать, что $f_n \rightarrow f$ почти всюду
	(потому что если доказать для $\sup$ по подпоследовательности, неравенство будет верно и для последовательности).
	Пусть $g_n = \min\left\{f_n, f\right\}$. Тогда $g_n \leqslant f$.
	Рассмотрим два случая:
	\begin{enumerate}
		\item 
			$f$ - суммируема.
			Тогда, по теореме Лебега, $\intl{E} g_n \rightarrow \intl{E} f$. Предел последовательности $\intl{E} g_n$ не превзойдет своего верхнего предела, 
			поэтому $\intl{E} f \leqslant \sup\limits_{n} \intl{E} g_n \leqslant \sup\limits_{n}\intl{E} f_n$
		\item
			$\intl{E} f = +\infty$. Тогда $\forall e$ - допустимо для $f$. $\intl{e} f < +\infty$
			Как мы показали, $\intl{e} f \leqslant \sup\limits_{n}\intl{E} f_n$. Переходя к $\sup$ по $e$ получаем необходимое неравенство.
	\end{enumerate}
\end{proof}

\newpage
