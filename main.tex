\documentclass{article}

\usepackage{amsmath}
\usepackage{wrapfig}
\usepackage{blindtext}
\usepackage{tikz}
\usepackage{yhmath}
\usepackage{amssymb}
\usepackage{amsthm}
\usepackage{mathtext}
\usepackage[T1,T2A]{fontenc}
\usepackage[utf8]{inputenc}
\usepackage[russian]{babel}
%\usepackage{geometry}
\usepackage[margin=2cm]{geometry}
\usepackage[mathscr]{euscript}
\usepackage{microtype}
\usepackage{bnf}
\usepackage{enumitem}
\usepackage{bm}
\usepackage{listings}
\usepackage{cancel}
\usepackage{proof}
\usepackage{epigraph}
\usepackage{titlesec}
\usepackage{mathtools}
%\setmainfont[Ligatures=TeX,SmallCapsFont={Times New Roman}]{Palatino Linotype}
\DeclareMathOperator{\grad}{grad}
\DeclareMathOperator{\Div}{div}
\DeclareMathOperator{\rot}{rot}
\DeclareMathOperator{\sign}{sign}
\selectlanguage{russian}
\pagenumbering{gobble}


\title{%
	Вопросы к экзамену
	по математическому анализу за 4 семестр}
\date{}

\begin{document}

\theoremstyle{definition}
\newtheorem*{definition}{Определение}
\theoremstyle{plain}
\newtheorem{theorem}{Теорема}[section]
\newtheorem{axiom}{Аксиома}
\newtheorem{lemma}[theorem]{Лемма}
\newtheorem{statement}[theorem]{Утверждение}
\newtheorem{nb}[theorem]{N. B.}
\newtheorem{corollary}[theorem]{Следствие}
\theoremstyle{remark}
\newtheorem*{example}{Пример}
\newtheorem{property}[theorem]{Свойство}


\newcommand{\todo}{\textsc{\textbf{TODO}}}
\newcommand{\abs}[1]{\left|#1\right|}
\newcommand{\norm}[1]{\left\|#1\right\|}
\newcommand{\normp}[1]{\norm{#1}_p}
\newcommand{\normpp}[2]{\norm{#1}_{#2}}
\newcommand{\intl}[1]{\int\limits_{#1}}
\newcommand{\defeq}{\mathrel{\stackrel{\makebox[0pt]{\mbox{\normalfont\tiny def}}}{=}}}
\makeatletter
\newcommand*{\rom}[1]{\expandafter\@slowromancap\romannumeral #1@}
\makeatother
\newcommand{\iintl}[1]{\iint\limits_{#1}}
\newcommand{\pdiff}[2]{\frac{\partial #1}{\partial #2}}
\newcommand{\intlr}[2]{\int\limits_{#1}^{#2}}
\newcommand{\suml}[1]{\sum\limits_{#1}}
\newcommand{\sumlr}[2]{\sum\limits_{#1}^{#2}}
\newcommand{\feps}{\forall\varepsilon}
\newcommand{\Epsilon}{\varepsilon}
\newcommand{\scalarp}[2]{\langle #1 , #2\rangle}
\newcommand{\set}[1]{\left\{#1\right\}}
\maketitle
\pagenumbering{gobble}
\begin{enumerate} \setlength\itemsep{0em}
	\item Критерий Лебега интгерируемоси по Риману
	\item Понятние суммируемой функции. $\sigma$-аддитивность и линейность несобственного интеграла Лебега для неотрицательной суммируемой функции.
	\item О распространении основных свойств интеграла Лебега на суммируемые функции произвольного знака.
	\item Теорема Лебега о мажорируемой сходимости.
	\item Теорема Леви.
	\item Теорема Фату.
	\item Полнота $L_p$.
	\item Неравенства Гельдера и Минковского.
	\item Мера подграфика.
	\item Теорема Фубини.
	\item Вычисление криволинейного интеграла \rom{1} рода.
	\item Вычисление криволинейного интеграла \rom{2} рода.
	\item Вычисление поверхностного интеграла \rom{1} рода.
	\item Вычисление поверхностного интеграла \rom{2} рода.
	\item Формула Грина.
	\item Теоремы Стокса и Остроградского-Гаусса.
	\item Определение ряда Фурье. Формулы Эйлера-Фурье.
	\item Ядро и интеграл Дирихле.
	\item Лемма Римана-Лебега.
	\item Принцип локализации Римана.
	\item Ядро и интеграл Фейера.
	\item Теорема Фейера.
	\item Следствие о двух пределах.
	\item Всюду плотность пространства $C$ в пространстве $L_p$.
	\item Теорема Фейера в $L_p$.
	\item Теорема Вейерштрасса в $L_p$.
	\item Теорема Дини.
	\item Следствие о четырех пределах.
	\item Полная вариация функции и ее аддитивность.
	\item О разложении функции ограниченной вариации в разность возрастающих функций.
	\item О разложении функции ограниченной вариации в ряд Фурье.
	\item Разложение в ряд Фурье функции $\sign(x)$.
	\item Разложение в ряд Фурье функции $\abs{x}$.
\end{enumerate}
\end{document}
