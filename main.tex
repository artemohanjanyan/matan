\documentclass{article}

\usepackage{amsmath}
\usepackage{wrapfig}
\usepackage{blindtext}
\usepackage{tikz}
\usepackage{yhmath}
\usepackage{amssymb}
\usepackage{amsthm}
\usepackage{mathtext}
\usepackage[T1,T2A]{fontenc}
\usepackage[utf8]{inputenc}
\usepackage[russian]{babel}
%\usepackage{geometry}
\usepackage[left=2cm,right=2cm,top=2cm,bottom=2cm,bindingoffset=0cm]{geometry}
\usepackage[mathscr]{euscript}
\usepackage{microtype}
\usepackage{bnf}
\usepackage{enumitem}
\usepackage{bm}
\usepackage{listings}
\usepackage{cancel}
\usepackage{proof}
\usepackage{epigraph}
\usepackage{titlesec}
\usepackage{mathtools}
%\setmainfont[Ligatures=TeX,SmallCapsFont={Times New Roman}]{Palatino Linotype}

\selectlanguage{russian}


\title{%
	Математический анализ 4 семестр \\
	\large Конспект лекций Додонова Н. Ю.}
\author{shared with $\heartsuit$ by artemZholus}
\date{}

\begin{document}

\theoremstyle{definition}
\newtheorem*{definition}{Определение}
\theoremstyle{plain}
\newtheorem{theorem}{Теорема}[section]
\newtheorem{axiom}{Аксиома}
\newtheorem{lemma}[theorem]{Лемма}
\newtheorem{statement}[theorem]{Утверждение}
\newtheorem{corollary}[theorem]{Следствие}
\theoremstyle{remark}
\newtheorem*{example}{Пример}
\newtheorem{property}[theorem]{Свойство}


\newcommand{\todo}{\textsc{\textbf{TODO}}}
\newcommand{\abs}[1]{\left|#1\right|}
\newcommand{\norm}[1]{\left\|#1\right\|}
\newcommand{\normp}[1]{\norm{#1}_p}
\newcommand{\normpp}[2]{\norm{#1}_{#2}}
\newcommand{\intl}[1]{\int\limits_{#1}}
\newcommand{\defeq}{\mathrel{\stackrel{\makebox[0pt]{\mbox{\normalfont\tiny def}}}{=}}}
\makeatletter
\newcommand*{\rom}[1]{\expandafter\@slowromancap\romannumeral #1@}
\makeatother
\newcommand{\iintl}[1]{\iint\limits_{#1}}
\newcommand{\intlr}[2]{\int\limits_{#1}^{#2}}
\newcommand{\suml}[1]{\sum\limits_{#1}}
\newcommand{\sumlr}[2]{\sum\limits_{#1}^{#2}}
\newcommand{\feps}{\forall\varepsilon}
\newcommand{\Epsilon}{\varepsilon}
\newcommand{\scalarp}[2]{\langle #1 , #2\rangle}
\newcommand{\set}[1]{\left\{#1\right\}}
\maketitle
\tableofcontents
%\newpage
\section{Критерий Лебега интегрируемости по Риману}

\begin{definition}[Колебание на отрезке]
	\[ \omega(f, c, d) = \sup\limits_{[c, d]} f - \inf\limits_{[c, d]} f =  
    \text{(по лемме из 1го семестра)}  = \sup\limits_{x',x'' \in [c,d]} | f(x') - f(x'') |\]
\end{definition}

\begin{definition}[Колебание функции в точке]
	\[ \omega(f, x) = \lim\limits_{\delta \rightarrow 0} \omega(f, x + \delta, x - \delta)\]
\end{definition}

Очевидно, колебание на отрезке неотрицательно, и, если $0 < \delta_1 < \delta_2$ 
то $\omega(f, x - \delta_1, x + \delta_1) < \omega(f, x - \delta_2, x + \delta_2)$.
Поэтому, вышеприведенный предел существует.

\begin{statement}
    $\omega(f, x) = 0 \Leftrightarrow f \in C(x)$
\end{statement}

\begin{proof}
	\begin{enumerate}
		\item $\Leftarrow$ Раз функция непрерывна, значит она достигает на отрезке своего $\sup$ и $\inf$. 
            Значит, если устремить границы отрезка к одной точке, в пределе получим разность двух одинаковых чисел.
		\item $\Rightarrow$ $\omega(f, x) = 0$ означает, что можно подобрать такую $\delta-$окрестность для $x$, 
            что она будет сколь угодно малой. Берем формулу $\sup\limits_{x',x'' \in [x - \delta,x + \delta]} | f(x') - f(x'') | = 0$ 
            фиксируем $x'' = x$ (от этого $\sup$ разве что уменьшится) и получаем определение непрерывности в $x$.
	\end{enumerate}
\end{proof}

\begin{definition}
$\tau:$ - разбиение отрезка $[a, b]$, если $\tau = \{x_j\}: \: a = x_0 < x_1 < \dots < x_n = b$
\end{definition}

Ведем кусочно-постоянную функцию $g(\tau, x) = \omega(f, x_j, x_{j + 1}),$ при $x \in [x_j, x_{j + 1}]$

\begin{statement}
$g(\tau_n, x) \xrightarrow[n \rightarrow +\infty]{} \omega(f, x)$ почти всюду на отрезке
\end{statement}

\begin{proof}
    Очевидно, мы можем подбирать $\tau_n$ так, чтобы границы отрезка, содержащего $x$ 
    совпали с границами из определения $\omega(f,x)$. Тогда для неграничных точек получим стремление. 
    Граничных точек на конечном шаге - конечное число, а это значит, что мы не перейдем за границу счетной 
    мощности (danger zone - МАТЛОГИКА), и предел будет почти всюду 
\end{proof}

Тогда, по теореме Лебега о предельном переходе под знаком интеграла, получаем:

\begin{gather*}
    \int\limits_{[a, b]}g(\tau_n, x)dx \rightarrow \int\limits_{[a,b]}\omega(f,x)dx
\end{gather*}

Левая часть, по лемме из первого семестра равна $\int\limits_{[a, b]}g(\tau_n, x)dx = \omega(f, \tau_n)$.
Получаем:

\begin{gather*}
    \lim\limits_{rang\tau_n \rightarrow 0} \omega(f, \tau_n) = \int\limits_{[a,b]} \omega(f,x)dx
\end{gather*}

Это наша рабочая формула.

\begin{theorem}[Критерий Лебега интегрируемости по Риману]
    $\\ f \in \Re (a, b) \Leftrightarrow \lambda\{ a : f \notin C(a) \} = 0$
\end{theorem}

\begin{proof}
	\begin{enumerate}
		\item 
			$\Rightarrow \\$ Пусть $\omega(f,x) = 0$ почти всюду на $[a, b]$. Тогда $\int\limits_{[a,b]} \omega(f, x)dx = 0 \: \Rightarrow f \in \Re [a,b]$ (Напрямую следует из утверждения $1.2$) 
		\item 
			$\Leftarrow \\$ Пусть $f \in \Re [a, b]$. Тогда, по определению, $\omega(f, \tau_n) \rightarrow 0$. 
            Тогда $\int\limits_{[a,b]} \omega(f, x)dx = 0$. Но $\omega(f, x) \geqslant 0$. 
            Значит $\omega(f,x) = 0$ почти всюду на $[a,b]$ (И, по лемме, почти всюду непрерывна).
	\end{enumerate}
\end{proof}




\section{Cуммируемые функции}
\subsection{Неотрицательные суммируемые функции}

Здесь и далее считаем, что мера $\mu$ - полная и $\sigma$-конечная.
Наша задача - распространить интеграл Лебега на более широкую ситуацию. Считаем, что $E \in \mathscr{A}$, $f: E \xrightarrow[]{измеримо}\mathbb{R}$, $f(x) \geqslant 0$ на $E$.

\begin{definition}
	$e \subset E$ называется допустимым для $f$ если:
	\begin{enumerate}
	\item
		$\mu(e) < +\infty$
	\item
		$f$ - ограничена на $e$
	\end{enumerate}
\end{definition}

\begin{statement}
	Непустые допустимые множества существуют.
\end{statement}

\begin{proof}
	Пусть $E_n = E(n < f(x) \leqslant n + 1)$. Понятно, что $E = \bigcup\limits_{n} E_n$. По $\sigma$-конечности $X = \bigcup\limits_{m}X_m$, причем $X_m$  - конечномерны. Тогда $E = \bigcup\limits_{n,m} E_nX_m$ - допустимые множества. Если они все пустые, то $E$, тоже пусто. Значит среди них хотя бы ожно непустое.
\end{proof}

\begin{definition}[Несобственный интеграл Лебега]
	\[\int\limits_{E} f d\mu \defeq \sup\limits_{e - допустимо} \int\limits_{e} fd\mu\]
\end{definition}

\begin{definition}[Неотрицательная суммируемая функция]
	Неотрицательная функция $f$ называется суммируемой на множестве $E$, если $\int\limits_{E} f d\mu < +\infty$
\end{definition}

Очевидно, если $\mu E < +\infty, \: f(x) \geqslant 0,$ то $\int\limits_{E} f d\mu = \sup\limits_{e \subset E} \int\limits_{e} fd\mu$.

Проверим аддитивность и линейность.

\begin{theorem}[$\sigma$-aддитивность несобственного интеграла Лебега]
	$\\*$Пусть $E = \bigcup\limits_{n} E_n$ - дизъюнктны. Тогда $\int\limits_{E} f = \sum\limits_{n} \int\limits_{E_n} f$
\end{theorem}

Докажем в два этапа. сначала конечную аддитивность, потом $\sigma$-аддитивность

\begin{proof}
	\begin{enumerate}
		\item 
			Пусть $E = E_1 \cup E_2$.Пусть $e_1 \in E_1$, $e_2 \in E_2$ - допустимые. И любое допустимое для $E$ множество $e = e_1 \cup e_2$.
			Для определенного интеграла мы знаем, что $\int\limits_{e} f = \int\limits_{e_1} f + \int\limits_{e_2} f \leqslant \int\limits_{E_1} f + \int\limits_{E_2} f$
			Переходя к $\sup$ по $e$ получаем $\int\limits_{E} f \leqslant \int\limits_{E_1} f + \int\limits_{E_2} f \\$
			В обратную сторону. Считаем, что $f$ - суммируема (иначе все тривиально). По определению $\sup$, $\feps > 0 \: \exists e_j \subset E_j : \int\limits_{E_j} f - \Epsilon < \int\limits_{e_j} f$. $\\$
			$\int\limits_{E_1} f + \int\limits_{E_1} f - 2\Epsilon < \int\limits_{e_1} f + \int\limits_{e_2} f = \int\limits_{e} f \leqslant \int\limits_{E} f $. Устремив $\Epsilon \rightarrow 0$ получим $\int\limits_{E_1} f + \int\limits_{E_2} f \leqslant \int\limits_{E} f. \\ $
			Значит $\int\limits_{E_1} f + \int\limits_{E_2} f  = \int\limits_{E} f $
		\item
			Итак, пусть $e = \bigcup\limits_{n = 1}^{+\infty} e_n$. Очевидно $\int\limits_{e_n} f \leqslant  \int\limits_{E_n} f$ и $ \int\limits_{e} f = \sum\limits_{n}\int\limits_{e_n} f$. Значит $ \int\limits_{E} f \leqslant  \sum\limits_{n} \int\limits_{E_n} f . \\$
			Обратно. $\feps > 0 \: \exists e_n \subset E_n : \\
			\int\limits_{E_n} f - \frac{\Epsilon}{2^n} <  \int\limits_{e_n} f $.  Сложим первые $p$ неравенств:$ \sum\limits_{1}^{p}\int\limits_{E_n} f - \Epsilon \sum\limits_{1}^{p} \frac{1}{2^n} < \sum\limits_{1}^{p}  \int\limits_{e_n} f \leqslant  \int\limits_{E} f$. Устремляя $p \rightarrow +\infty$, получаем $ \sum\limits_{n = 1}^{+\infty}\int\limits_{E_n} f - \Epsilon \leqslant  \int\limits_{E} f$. Теперь устремим $\Epsilon \rightarrow 0$ и получим обратное неравенство.
	\end{enumerate}
\end{proof}

\newpage

\begin{theorem}[Линейность несобственного интеграла Лебега]
	$\\*$
	\begin{enumerate}
		\item
			 $ \int\limits_{E} \alpha f  = \alpha  \int\limits_{E} f, \:\: \alpha > 0$
		\item
			 $ \int\limits_{E} (f + g) =  \int\limits_{E} f +  \int\limits_{E} g$
	\end{enumerate}
\end{theorem}

\begin{proof}
	Первое свойство следует непосредственно из определения. Докажем второе.
	Итак, пусть $E_n = E(n < f + g \leqslant n + 1)$. Тогда, очевидно, $E = \bigcup\limits_{n} E_n$. По $\sigma$-конечности можно написать $X = \bigcup\limits_{n} X_n$.
	От $X_n$ мы хотим дизъюнктности, поэтому, если они не таковы, то проделаем следующий трюк: $\\* X = X_1 \cup (X_2 \setminus X_1) \cup \dots \cup (X_n \setminus \bigcup\limits_{1}^{n-1}X_j) \cup \dots$.
	Теперь  $E$ можно разбить как $E = \bigcup\limits_{n, m} E_n X_m$ - эти множества дизъюнктны и допустимы для $f + g$. Далее по $\sigma$-аддитивности пишем: 
	$ \int\limits_{E} (f + g) = \sum\limits_{n}  \int\limits_{A_n} (f + g) = $ (по линейности определенного интеграла) $ =  \sum\limits_{n} \int\limits_{A_n} f + \sum\limits_{n} \int\limits_{A_n} g = $(по $\sigma$-аддитивности несобственного)$ = \int\limits_{E} f + \int\limits_{E} g$
\end{proof}

\begin{statement}
	Если $0 \leqslant f \leqslant g$, то $\int\limits_{E} f \leqslant \int\limits_{E} g$
\end{statement}

\begin{proof}
	$0 \leqslant g - f$ - по арифметике измеримости, эта функция суммируема. Раз она неотрицательна, интеграл от нее тоже. \[0 \leqslant \int\limits_{E} g - f =  \int\limits_{E} g -  \int\limits_{E} f \: \Rightarrow  \int\limits_{E} f \leqslant  \int\limits_{E} g\]
\end{proof}

\subsection{Суммируемые функции произвольного знака}

\begin{definition}
	\[
	f^+(x) = 
	\begin{cases}
		0 &, f(x) < 0 \\
		f(x) &, f(x) \geqslant 0
	\end{cases}
	\]
	$\\*$
	\[
	f^-(x) = 
	\begin{cases}
		-f(x) &, f(x) < 0 \\
		0 &, f(x) \geqslant 0
	\end{cases}
	\]
\end{definition}

Заметим, что $f = f^+ - f^-$, $|f| = f^+ + f^-$. $f^+$ и $f^-$ - неотрицательные суммируемые функции (если $f$ - измерима).


\begin{definition}
	$f$ называется суммируемой на $E$, если одновременно $f^+$ и $f^-$ - суммируемы.
	\[ \int\limits_{E} f \defeq  \int\limits_{E} f^+ -  \int\limits_{E} f^- \]
\end{definition}

\begin{statement}
	$f$ - суммируема $\Leftrightarrow$ $|f|$ - суммируема.
\end{statement}

\begin{proof}
	$f$ - суммируема тогда и только тогда когда $f^+$ и $f^-$ - суммируемы. $|f|$ - суммируема тогда и только тогда, когда $f^+$ и $f^-$ - суммируемы. 
\end{proof}

Проверим $\sigma$-аддитивность и линейность для случая функции произвольного знака:

\newpage

\begin{theorem}[Аддитивность в сдучае произвольного знака]
	Пусть $E = \bigcup\limits_{n} E_n$ - дизъюнктные, тогда $\int\limits_{E} f = \sum\limits_{n} \int\limits_{E_n} f$
\end{theorem}

\begin{proof}
	$\int\limits_{E} f^+ = \sum\limits_{n} \int\limits_{E_n} f^+$, то же для $f^-$. Тогда $\int\limits_{E} f = \int\limits_{E} f^+ +  \int\limits_{E} f^- = \sum\limits_{n}\int\limits_{E_n} f^+ + \sum\limits_{n}\int\limits_{E_n} f^- = \sum\limits_{n} (\int\limits_{E_n} f^+ - f^-) = \sum\limits_{n}\int\limits_{E_n} f $
\end{proof}


\begin{theorem}[Линейность в случае произвольного знака]
        $\\*$
        \begin{enumerate}
                \item
			$ \int\limits_{E} \alpha f  = \alpha  \int\limits_{E} f, \: \alpha \in \mathbb{R}$
                \item
                         $ \int\limits_{E} (f + g) =  \int\limits_{E} f +  \int\limits_{E} g$
        \end{enumerate}
\end{theorem}

\begin{proof}
	Пункт 1 очевиден, не будем на нем останавливаться. Докажем пункт 2:$\\*$
	\[
		\int\limits_{E} f + \int\limits_{E} g = (\int\limits_{E} f^+ \int\limits_{E} g^+) - (\int\limits_{E} f^- + \int\limits_{E} g^-) = \int\limits_{E} (f^+ + g^-) - \int\limits_{E} (f^- + g^-) = (\ast) = \int\limits_{E} (f + g)^+ - \int\limits_{E} (f + g)^- = \int\limits_{E} (f + g)
	\]
	Проверим переход  $(\ast)$. Для этого нужно, чтобы выполнялось $(f^+ + g^+) = (f + g)^+$ - в общем случае, это неправда. Поэтому нужно рассмотреть много случаев:
	\begin{enumerate}
		\item
			$f \geqslant 0, \: g\geqslant 0 \Rightarrow$ пусть $E_1 = E(f \geqslant 0, \: g \geqslant 0)$
		\item
			$ f \leqslant 0, \: g \leqslant 0 \Rightarrow$ пусть $E_2 = E(f \leqslant 0, \: g \leqslant 0)$
		\item
			$f \geqslant 0, \: g\leqslant 0 \Rightarrow$ тут нужно различить два подслучая:
			\begin{enumerate}
				\item 
					$f + g \geqslant 0 \Rightarrow$ пусть $E_3 = E(f \geqslant 0, \: g\leqslant 0, f + g \geqslant 0)$
				\item
					$f + g < 0 \Rightarrow$ пусть $E_4 = E(f \geqslant 0, \: g\leqslant 0, f + g < 0)$
			\end{enumerate}
		\item
			$f \leqslant 0, \: g\geqslant 0 \Rightarrow$ аналогично, два подслучая:
			\begin{enumerate}
				\item 
					$f + g \geqslant 0 \Rightarrow$ пусть $E_5 = E(f \leqslant 0, \: g\geqslant 0, f + g \geqslant 0)$
				\item
					$f + g < 0 \Rightarrow$ пусть $E_6 = E(f \leqslant 0, \: g\geqslant 0, f + g < 0)$
			\end{enumerate}
	\end{enumerate}
	Очевидно, эти множества дизъюнктны (на $0$ забьем) и можно написать: $\int\limits_{E} f = \sum\limits_{1}^{6}\int\limits_{E_j} f$.
	Дальше идет нудный разбор случаев, я потом напишу \todo
\end{proof}



\section{Предельный переход в классе суммируемых функций}

\subsection{Теорема Лебега о мажорируемой сходимости}

\begin{theorem}[Теорема Лебега о мажорируемой сходимости]
	Пусть $f_n \Rightarrow f$ на $E$, $|f_n| \leqslant \phi$ на $E$, $\phi$ - суммируема.
	$\\$ Тогда:
	\begin{enumerate}
		\item
			$f$ - суммируема
		\item
			$\intl{E} f_n \rightarrow \intl{E} f$
	\end{enumerate}
\end{theorem}

\begin{center}
	\framebox[4in]{
		\begin{minipage}[t]{3.5in}% я дезигнер
			Следует иметь ввиду, что в условии теоремы достаточно требовать выполнения свойств почти всюду.
		\end{minipage}
	}	
\end{center}

\begin{theorem}
	Пусть $f$ - суммируема на $E$. Тогда $\feps > 0 \: \exists \delta > 0 : \forall E' \subset E \Rightarrow \mu E' < \delta \Rightarrow |\intl{E'} f| < \Epsilon$ 
\end{theorem}

\begin{proof}
	По определению, можно написать $\feps > 0 \: \exists e : \intl{E \setminus e} |f| < \Epsilon$. Так как $e$ - допустимо, $f$ - ограничена на $e$ и $E = (E \setminus e) \cup e$. Возьмем любое $E' \subset E$, тогда $E' = E'(E \setminus e) \cup E'e$.
	\[
		\abs{\intl{E'} f} \leqslant \abs{\intl{E'(E \setminus e)} f} + \abs{\intl{E'e} f} \leqslant \Epsilon + \abs{\intl{E'e} f}
	\]
	Мы считаем, что $\abs{f(x)} \leqslant M$. Заметим, что выбор $E'$ не накладывал никаких ограничений на $M$. Тогда:
	\[
		\intl{E'e} \abs{f} \leqslant M \mu E'e \leqslant M \mu E'
	\]
	Поэтому $\delta$ мы можем выбрать как $\delta = \frac{\Epsilon}{M}$. И получится, что $\mu E' \leqslant \delta \Rightarrow \abs{\intl{E'} f} \leqslant 2\Epsilon$
\end{proof}

\begin{proof}[Доказательство теоремы Лебега]
	По теореме Рисса $f_{n_k} \rightarrow f$ почти всюду, причем $\abs{f_{n_k}(x)} \leqslant \phi(x)$, занчит $\abs{f(x)} \leqslant \phi(x) \: \Rightarrow f -$ суммируема. 	Рассмотрим $\abs{\intl{E} f_n  - \intl{E} f} \leqslant \intl{E} \abs{f_n - f}$. Так как $\phi $ - суммируема, $\\* \feps > 0 \:\: \exists e \text{(допустимое для $\phi$)} : \intl{E \setminus e} \phi \leqslant \Epsilon$ 
	\[
		\intl{E} \abs{f_n - f} = \intl{E \setminus e} \abs{f_n - f} + \intl{e} \abs{f_n - f} \leqslant 2\Epsilon + \intl{e} \abs{f_n - f}
	\]
	Пусть $\abs{\phi} \leqslant M \Rightarrow \abs{f_n - f} \leqslant 2M$. Так же мы знаем, что $\intl{e} \abs{f_n - f} \xrightarrow[n \rightarrow +\infty]{} 0$. Значит, начиная с некоторого $N_0$, $\intl{e} \abs{f_n - f} < \Epsilon$.
	Следовательно, начиная с $N_0$, $\intl{E} \abs{f_n - f} \leqslant 3\Epsilon$
\end{proof}

\newpage

\subsection{Теорема Леви}

\begin{theorem}[Теорема Леви]
	Пусть $f_n(x) \leqslant 0$, $f_n(x) \leqslant f_{n + 1}(x)$, $f(x) = \lim\limits_{n\rightarrow +\infty} f_n(x)$ на $E$. 
	Тогда $\intl{E} f_n \rightarrow \intl{E} f$
\end{theorem}


\begin{proof}
	Два случая:
	\begin{enumerate}
		\item
			$f$ - почти всюду конечна на $E$.
			Два подслучая:
			\begin{enumerate}
				\item
					$\intl{E} f < +\infty$. Так как $\abs{f_n(x)} \leqslant f(x) \: \Rightarrow$ $f$ - суммируемая мажоранта для $f_n$, и теорема верна по теореме Лебега о мажорируемой сходимости.
				\item
					$\intl{E} f = +\infty$. ($f$ все еще мажоранта для $f_n$, но уже не суммируемая) Мы поступим так. Раз $\sup\limits_{e - допустимо} \intl{e} f = +\infty$, значит $\forall c > 0 \: \exists e - допустимое для f : c < \intl{e} f$. В силу $f_n \leqslant f$  по теореме Лебега о мажорируемой сходимости $\intl{e} f_n \rightarrow \intl{e} f$. Это значит, начиная с некоторого $N_0$, $c < \intl{e} f_n \leqslant \intl{E} f_n \: \Rightarrow \intl{E} f_n \rightarrow +\infty = \intl{E} f$
			\end{enumerate}
		\item
			$\mu E(f = +\infty) > 0$ (Расслабьтесь, и будет не больно)$\\$ 
			Очевидно, в этой ситуации может быть только $\intl{E} f = +\infty$. Из $f_n(x) \leqslant f_{n + 1}(x) \: 
			\Rightarrow \intl{E} f_{n}(x) \leqslant \intl{E} f_{n + 1}(x)$. 
			По теореме Вейерштрасса, у последовательности $\left\{ \intl{E} f_n \right\}$ будет существовать предел. 
			Причем он будет конечным тогда и только тогда, когда эта последовательность ограничена. 
			Так что нам нужно вывести противоречие из того факта, что эта последовательность ограничена.
			Предположим, что это так: пусть $\intl{E} f_n \leqslant M$. Итак, зафиксируем $\forall c > 0$. Рассмотрим $E(f_n \geqslant c) \subset E$.
			\begin{gather*}
				\intl{E(f_n \geqslant c)} f_n \leqslant M \\ c \mu E(f_n \geqslant c) \leqslant \intl{E(f_n \geqslant c)} f_n \Rightarrow 
				\mu E(f_n \geqslant c) \leqslant \frac{M}{c}
			\end{gather*}
			Можно проверить, что:
			\begin{gather*}
				E(f = +\infty) \subset \bigcup\limits_{m = 1}^{\infty} \bigcap\limits_{n = m}^{\infty} E(f_n \geqslant c)
			\end{gather*}
			\begin{proof}
				Пусть $x \in E(f = +\infty)$. Значит $f_n(x) \xrightarrow[n \rightarrow +\infty]{} +\infty$. 
				Следовательно $\forall c > 0 \exists N_x : \forall n > N_x \Rightarrow f_n(x) \geqslant c \xRightarrow[def]{by} 
				x \in \bigcap\limits_{n = N_x}^{\infty} E(f_n \geqslant c)$
			\end{proof}
			Заметим одну интересную штуку. 
			\begin{gather*}
				\forall c > 0 f_n(x) \geqslant c \Rightarrow f_{n + 1}(x) \geqslant c \Rightarrow 
				E(f_n \geqslant c) \subset E(f_{n + 1} \geqslant c) \Rightarrow \bigcap\limits_{n = m}^{\infty} E(f_n \geqslant c) = E(f_m \geqslant c)
				\Rightarrow \\
				\Rightarrow \bigcup\limits_{m = 1}^{\infty} \bigcap\limits_{n = m}^{\infty} E(f_n \geqslant c) = 
				\lim\limits_{m \rightarrow +\infty} E(f_m \geqslant c)
			\end{gather*}
			Отсюда делаем вывод, что: 
			\begin{gather*}
				\mu \bigcap\limits_{n = m}^{\infty} E(f_n \geqslant c) \xrightarrow[m \rightarrow +\infty]{} 
				\mu \bigcup\limits_{m = 1}^{\infty} \bigcap\limits_{n = m}^{\infty} E(f_n \geqslant c) \geqslant
				\mu E(f = +\infty)
			\end{gather*}
			\newpage
			\begin{gather*}
				\mu \bigcap\limits_{n = m}^{\infty} E(f_n \geqslant c) = \mu E(f_m \geqslant c) \leqslant \frac{M}{c} \Rightarrow \\
				\Rightarrow \mu \bigcup\limits_{m = 1}^{\infty} \bigcap\limits_{n = m}^{\infty} E(f_n \geqslant c) \leqslant \frac{M}{c} \Rightarrow \\
				\Rightarrow \mu E(f = +\infty) \leqslant \frac{M}{c}
			\end{gather*}
			$c$ - любое, поэтому можно устремить $c \rightarrow +\infty$. Значит $\mu E(f = +\infty) = 0$. Противоречие получено.
	\end{enumerate}
\end{proof}

\begin{corollary}
	Пусть $u_n(x) \geqslant 0$ и $\suml{n}\intl{E}u_n$ - сходится. Тогда $\suml{n}u_n(x)$ - сходится почти всюду на $E$.
\end{corollary}

\begin{proof}
	$S_n = u_1 + u_2 + \dots + u_n$. Так как интеграл сходится, его частичная сумма ограничена. $M \geqslant \sumlr{n = 1}{m}\intl{E} u_n = \intl{E} S_m$		
	Обозначим $S(x) = \suml{n} u_n(x)$. В силу неотрицательности $u_n(x)$, $S_n(x)$ - возрастает ($S_n \leqslant S_{n + 1}$), и 
	$S(x) = \lim\limits_{n \rightarrow +\infty} S_n(x)$. Следовательно, по теореме Леви $\intl{E} S_n \rightarrow \intl{E} S$. Следовательно, $S$ - суммируемая функция, это значит, что она почти всюду конечна. 
\end{proof}

\begin{corollary}
	Пусть $f \geqslant 0,  \:\: f_n(x) = \min\left\{f(x), n\right\}$ - срезка функции $f$. Тогда $\intl{E} f_n \rightarrow \intl{E} f$.
\end{corollary}

\begin{proof}
	$f_n$ удовлетворяют условиям теоремы Леви.
\end{proof}

\subsection{Теорема Фату}

\begin{theorem}[Теорема Фату]
	Пусть $f_n \geqslant 0$, $f_n \Rightarrow f$ на $E$. Тогда \[\intl{E} f \leqslant \sup\limits_{n \in \mathbb{N}} \intl{E} f_n	\]
\end{theorem}

\begin{proof}
	Применим теорему Рисса, получив, что $f_{n_k} \rightarrow f$ почти всюду. Без ограничения общности можем считать, что $f_n \rightarrow f$ почти всюду
	(потому что если доказать для $\sup$ по подпоследовательности, неравенство будет верно и для последовательности).
	Пусть $g_n = \min\left\{f_n, f\right\}$. Тогда $g_n \leqslant f$.
	Рассмотрим два случая:
	\begin{enumerate}
		\item 
			$f$ - суммируема.
			Тогда, по теореме Лебега, $\intl{E} g_n \rightarrow \intl{E} f$. Предел последовательности $\intl{E} g_n$ не превзойдет своего верхнего предела, 
			поэтому $\intl{E} f \leqslant \sup\limits_{n} \intl{E} g_n \leqslant \sup\limits_{n}\intl{E} f_n$
		\item
			$\intl{E} f = +\infty$. Тогда $\forall e$ - допустимо для $f$. $\intl{e} f < +\infty$
			Как мы показали, $\intl{e} f \leqslant \sup\limits_{n}\intl{E} f_n$. Переходя к $\sup$ по $e$ получаем необходимое неравенство.
	\end{enumerate}
\end{proof}

\newpage

\section{Пространства $L_p$}

\begin{definition}
	$L_p(E) \defeq \left\{f : E \rightarrow \mathbb{R}, f - \text{измерима}, \intl{E} \abs{f}^p < +\infty \right\}$
\end{definition}

Нам нужно проверить, что $L_p(E)$ - НП. То есть, $f,g \in L_p \Rightarrow \alpha f + \beta g \in L_p$, $\norm{f}_p = \left( \intl{E} \abs{f}^p \right)^{\frac{1}{p}}$.
Причем, $\norm{f}$ - удовлетворяет аксиомам нормы. 

\begin{statement}
	$\normp{f}$ удовлетворяет двум свойствам:
	\begin{enumerate}
		\item
			$\normp{\alpha f } = |\alpha| \normp{f}$
		\item
			$\normp{f + g} \leqslant \normp{f} + \normp{g}$
	\end{enumerate}
\end{statement}

\begin{proof}
	\begin{enumerate}
		\item
			Очевидно.
		\item
			$\intl{E} \abs{f + g}^p \leqslant \intl{E} (\abs{f} + \abs{g})^p$. Пусть $E_1 = E(\abs{f} \leqslant \abs{g}), \: E_2 = E(\abs{f} > \abs{g})$.
			Тогда $E = E_1 \cup E_2$. 
			\begin{gather*}
				\intl{E} (\abs{f} + \abs{g})^p = \intl{E_1} (\abs{f} + \abs{g})^p + \intl{E_2} (\abs{f} + \abs{g})^p \leqslant \\
				\leqslant \intl{E_1} (2\abs{g})^p + \intl{E_2} (2\abs{f})^p < +\infty
			\end{gather*}
			Следовательно $f + g \in L_p$
	\end{enumerate}
\end{proof}

\begin{theorem}[Неравенство Гельдера]
	Пусть $p > 1$ и $q : \frac{1}{p} + \frac{1}{q} = 1$. Пусть $f \in L_p, \: g \in L_q$. Тогда
	\begin{gather*}
		\intl{E} \abs{f}\abs{g} \leqslant \left( \intl{E} \abs{f}^p \right)^{\frac{1}{p}}  \left( \intl{E} \abs{g}^q \right)^{\frac{1}{q}}
	\end{gather*}
\end{theorem}


\begin{proof}
	Воспользуемся неравенством Юнга: ($uv \leqslant \frac{1}{p}u^p + \frac{1}{q}v^q$).
	Пусть $u =  \frac{\abs{f}}{\normp{f}}$, $v = \frac{\abs{g}}{\normp{g}}$
	\begin{gather*}
		\frac{\abs{f}\abs{g}}{\normp{f}\normp{g}} \leqslant \frac{1}{p} \frac{\abs{f}^p}{\normp{f}^p} + \frac{1}{q} \frac{\abs{g}^q}{\normp{g}^q}\\
		\intl{E}\frac{\abs{f}\abs{g}}{\normp{f}\normp{g}} \leqslant 
		\frac{1}{p} \intl{E}\frac{\abs{f}^p}{\normp{f}^p} + 
		\frac{1}{q} \intl{E}\frac{\abs{g}^q}{\normp{g}^q} = \frac{1}{p} + \frac{1}{q} = 1 
	\end{gather*}
\end{proof}

\begin{theorem}[Неравенство Минковского]
	Пусть $p > 1$, $f, g \in L_p$. Тогда
	\begin{gather*}
		\left(\intl{E} (\abs{f} + \abs{g})^p \right)^{\frac{1}{p}} \leqslant 
		\left( \intl{E} \abs{f}^p \right)^{\frac{1}{p}} + 
		\left( \intl{E} \abs{g}^p \right)^{\frac{1}{p}}
	\end{gather*}
\end{theorem}

\begin{proof}
	Рассмотрим $(f + g)^p = f(f + g)^{p - 1} + g(f + g)^{p - 1}$.
	\begin{gather*}
		\intl{E} (f + g)^p = \intl{E} f(f + g)^{p - 1} + \intl{E}g(f + g)^{p - 1} \leqslant \\
		\leqslant \left( \intl{E} f^p \right)^{\frac{1}{p}} \left(\intl{E} (f + g)^{q(p - 1)} \right)^{\frac{1}{q}} + 
	  		  \left( \intl{E} g^p \right)^{\frac{1}{p}} \left(\intl{E} (f + g)^{q(p - 1)} \right)^{\frac{1}{q}} \\
		\text{пусть}\: q = \frac{p}{p - 1} \\
		\left(\intl{E}(f + g)^p \right)^{1 - \frac{1}{q}} \leqslant  \left( \intl{E} f^p \right)^{\frac{1}{p}} +  \left( \intl{E} g^p \right)^{\frac{1}{p}} \\
		\frac{1}{p} = 1 - \frac{1}{q}
	\end{gather*}
\end{proof}

Если подставить в неравенство Минковского определение нормы, то можно заметить, что мы доказали неравенство треугольника.

\begin{theorem}
	$L_p(E) $ - Банахово пространство.
\end{theorem}

Докажем вспомогательную лемму:

\begin{lemma}
	Пусть $f_n$ - измеримы, и $\forall\delta > 0\: \mu E\left(\abs{f_n - f_m} \geqslant \delta\right) \xrightarrow[n,m \rightarrow +\infty]{} 0$.
	Тогда $\exists n_1 < n_2 < \dots < n_k < \dots : f_{n_k} \rightarrow f$ почти всюду.
\end{lemma}

\begin{proof}
	Пусть $\delta = \frac{1}{2^k}$. Можно проверить, что (\todo) 
	$\exists n_1 < n_2 < \dots < n_k < \dots: \\$
	$\mu E\left(\abs{f_{n_{k + 1}} - f_{n_k}} \geqslant \frac{1}{2^k}\right) \leqslant \frac{1}{2^k}$.
	Рассмотрим следующее множество:
	\begin{gather*}
		E' = \bigcup\limits_{k = 1}^{\infty}\bigcap\limits_{j = k}^{\infty} E\left(\abs{f_{n_{j + 1}} - f_{n_j}} \leqslant \frac{1}{2^j} \right)
	\end{gather*}
	Рассмотрим функциональный ряд $S = f_1 + (f_2 - f_1) + (f_3 - f_2) + \dots$. Фиксируем $x \in E'$. Тогда 
	$\\ \exists k_x : x \in \bigcap\limits_{j = k_x}^{\infty} E\left(\abs{f_{n_{j + 1}} - f_{n_j}} \leqslant \frac{1}{2^j} \right)$.
	Это значит, что при $j > k_x$ выполняется $\abs{f_{n_{j + 1}}(x) - f_{n_k}(x)} \leqslant \frac{1}{2^j} \xrightarrow[j \rightarrow +\infty]{} 0$.
	Следовательно, на $E'$ функциональный ряд $S$ - сходится. 
	Нам осталось проверить, что его дополнение нуль-мерно.
	Т. е. $\mu \overline{E'} = 0$. Очевидно:
	\begin{gather*}
		\overline{E'} = \bigcap\limits_{k = 1}^{\infty}\bigcup\limits_{j = k}^{\infty} E\left(\abs{f_{n_{j + 1}} - f_{n_j}} > \frac{1}{2^j} \right) \Rightarrow \\
		\Rightarrow \overline{E'} \subset \bigcup\limits_{j = k}^{\infty} E\left(\abs{f_{n_{j + 1}} - f_{n_j}} > \frac{1}{2^j} \right) \Rightarrow \\
		\Rightarrow \mu \overline{E'} \leqslant \sumlr{j = k}{\infty} \mu E(\abs{f_{n_{j + 1}} - f_{n_j}} > \frac{1}{2^j}) \leqslant 
		\sumlr{j = k}{\infty} \frac{1}{2^j} \xrightarrow[k \rightarrow +\infty]{} 0 \Rightarrow \\
		\Rightarrow \mu \overline{E'} = 0
	\end{gather*}
\end{proof}

\newpage

\begin{proof}[Доказательство Теоремы]
\end{proof}

Может показаться, что требование измеримости функции $f$ в определении пространства $L_p$ -  излишне. Это отнюдь не так.
\begin{statement}
	Существует функция $f$ такая, что ее $p$-я степень измерима, но сама функция - нет. 
\end{statement}

\begin{proof}
	Рассмотрим произвольное неизмеримое множество $C \subset \mathbb{R}$. Тогда пусть
	\begin{gather*}
		f(x) = 
		\begin{cases}
			1  &, x \in C \\
			-1 &, x \notin C
		\end{cases}
	\end{gather*}
	Очевидно, $f$ -неизмерима (так как множество Лебега $E(f > \frac{1}{2}) = С$ - неизмеримо). Но $f^2(x) = 1$ при $x \in \mathbb{R}$ - очевидно, измеримая функция.
\end{proof}

\section{Мера подграфика}

Итак, рассмотрим $\left(X, \mathscr{A}, \mu \right)$. Считаем, что мера - полная и $\sigma$-конечная. 
$f: E \xrightarrow[]{\text{изм.}} \mathbb{R}$, $f \geqslant 0$ почти всюду.
\begin{definition}
	$G_f \defeq \left\{ (x,y) : x \in E, 0 \leqslant y \leqslant f(x) \right\}$ - подграфик функции $f$.
\end{definition}

Здесь и далее, в качестве $X \equiv \mathbb{R}^{n}, \: \mu \equiv \lambda_n$.

\begin{theorem}[Об измеримости подграфика]
	Подграфик измерим, и его мера равна $\lambda_{n + 1}(G_f) = \intl{E} fdx$
\end{theorem}

\begin{statement}
	$G_c(E)$ - измеримо, $\lambda G_c(E) = c \lambda E$, где $c$ - константа.
\end{statement}

\begin{proof}
	Пойдем от простого к сложному. Для ячейки $\mathbb{R}^n$ формула верна по определению. 
	Пусть теперь $E$ - открытое множество. Как известно, любое открытое множество представляется в виде $E = \bigcup\limits_{m} \Pi_m$, $\Pi_m$ - дизъюнктные ячейки.
	$G(E) = \bigcup\limits_{m} G(\Pi_m) \Rightarrow \lambda G(E) = \sum\limits_{m} \lambda G(\Pi_m) = c \sum\limits_{m}\lambda G(\Pi_m) = c\lambda E$.
	Далее, без ограничения общности, можем считать, что $\mu E < +\infty$ (Потому что у нас есть $\sigma$ - конечность; 
	$\\ \mathbb{R}^n = \bigcup\limits_{m}T_m \:\:(T_m : \lambda T_m < +\infty)\Rightarrow E = \bigcup\limits_{m} ET_m \:\: (\lambda ET_m < +\infty)$ ).
	Воспользуемся формулой: $\lambda^{\ast} E = \inf\limits_{E\subset G - \text{открыто}} \lambda G$
	По аксиоме выбора,  $\exists G_m : G_m \subset G_{m + 1}, E = \bigcap\limits_{m} G_m$. Понятно, что тогда $\lambda G_m \rightarrow \lambda E$. 
	Так же $G(E) = \bigcap\limits_{m} G(G_m)$. Следовательно $\lambda G(G_m) =  c\lambda G_m \rightarrow c\lambda E$
\end{proof}

\begin{proof}[Доказательство теоремы]
	Мы умеем писать суммы Лебега-Дарбу: $\underline{S}(\tau) \leqslant \intl{E} f \leqslant \overline{S}(\tau)$. 
	Важно, что интеграл Лебега - единственное число, которое обладает таким свойством. $\tau : E = \bigcup\limits_{m} e_m$ - конечное объединение дизъюнктных множеств, и	
	\begin{gather*}
		\underline{S}(\tau) = \sum\limits_{p} m_p \lambda e_p, \:m_p = \inf\limits_{x \in e_p} f(x) \\
		\overline{S}(\tau) = \sum\limits_{p} M_p \lambda e_p, \:M_p = \sup\limits_{x \in e_p} f(x)
	\end{gather*}
	Обозначим $\underline{E}_p = G_{m_p}(e_p)$. Тогда $\lambda \underline{E}_p = m_p \lambda e_p$. Пусть $\underline{E}(\tau) = \bigcup\limits_{p}\underline{E}_p$.
	Заметим, что 
	\begin{gather*}
		\lambda\underline{E}(\tau) = \sum\limits_{p = 1}^{n}\lambda\underline{E}_p = \sum\limits_{p = 1}^{n} m_p \lambda e_p = \underline{S}(\tau) \\
		\lambda\overline{E}(\tau) = \sum\limits_{p = 1}^{n}\lambda\overline{E}_p = \sum\limits_{p = 1}^{n} M_p \lambda e_p = \overline{S}(\tau) \\
		\underline{E}(\tau) \subset G_f(E) \subset \overline{E}(\tau)
	\end{gather*}
	По свойствам сумм Лебега-Дарбу: $\feps > 0\:\exists \tau_{\Epsilon}: \forall \tau \leqslant \tau_{\Epsilon}\: 
	\overline{S}(\tau) - \underline{S}(\tau) \leqslant \Epsilon \\$
	Сопоставляя это с предыдущими фактами, получаем $\underline{S}(\tau) \leqslant \lambda G_f(E) \leqslant \overline{S}(\tau).\\$
	И тогда, необходимо, $\lambda G_f(E) = \intl{E} f$
 
\end{proof}

\newpage

\section{Теорема Фубини}

\section{О многократных интегралах Римана}

Обобщим понятние интеграла Римана на множества большей размерности 
(для упрощения будем вести речь в терминах $\mathbb{R}^2$).
Итак, рассмотрим $\Pi = [a, b] \times [c, d]$, $\tau_1 : a = x_0 < x_1 < \dots < x_n = b$, 
$\tau_2 : c = y_0 < y_1 < \dots < y_n = d$. 
Тогда $\tau = \tau_1 \times \tau_2$, и 
$\Pi_{ij} = [x_i, x_{i + 1}) \times [y_j, y_{j + 1})$, $\Pi \equiv \bigcup\limits_{i, j} \Pi_{ij}$.
Теперь мы можем составить суммы Римана:
\begin{gather*}
	\sigma(f, \tau) = \suml{ij} f(\overline{x}_i, \overline{y}_j)\Delta x_i	\Delta y_j
\end{gather*}

Положим $rang\tau \equiv \max\limits_{ij}\left\{ diam \Pi_{ij}\right\}$.
Тогда:
\begin{gather*}
	\lim\limits_{rang\tau\rightarrow +\infty} \sigma(f, \tau) = \intlr{a}{b}\intlr{c}{d}f(x, y)dxdy
\end{gather*}
Если вышеприведенный предел существует , то он называется двойным интегралом Римана. 
Ясно, что функция интегрируемая по Риману на прямоугольнике интегрируема по Лебегу. 
Это позволяет воспользоваться теоремой Фубини:
\begin{gather*}
	\intlr{a}{b}f(x, y)dxdy = \intlr{a}{b}dx\intlr{c}{d}f(x, y)dy
\end{gather*}
В общем случае, это не означает, что $\intlr{a}{b}f(x, y)dy$ - интегрируема по Риману.
Далее, встает вопрос - как обобщить кратный интеграл на произвольное плоское множество?
Можно воспользоваться двумя равносильными подходами:
\begin{enumerate}
	\item 
		Пусть $\overline{f}: E \rightarrow \mathbb{R}$
		\begin{gather*}
			\overline{f} =  \left\{\begin{matrix}
				0,& (x, y) \notin E
									\\ 
										f(x, y), &\text{otherwise}
									\end{matrix}\right.
		\end{gather*}
		Так как $E$ - ограничено, его можно поместить в прямоугольник $\Pi$ и тогда:
		\begin{gather*}
			\iint\limits_{E}f \defeq \iint\limits_{\Pi} \overline{f}
		\end{gather*}
		Если $f \equiv 1$ на $E$ то тогда
		$\iint\limits_{E}f = \lambda E$
\end{enumerate}


\section{Криволинейные интегралы}

\subsection{Определение}

Интегралы, которые будут рассмотрены в данном параграфе будут частными случаями интегралов по многообразиям от дифференциальных форм.

Итак, рассмотрим кривую $\Gamma : \langle a, b \rangle \rightarrow \mathbb{R}^3$. 
Здесь и далее считаем корринатные функции непрерывно-дифференцируемыми, а дугу - спрямляемой 
(на всякий случай: эти понятия эквивалентны).
Напомним, что спрямляемость означает существование интеграла $l(\Gamma) = \intlr{a}{b} \sqrt{x'^2 + y'^2 + z'^2}dt$

Мы будет рассматривать 2 случая:
\begin{enumerate}
	\item
		$f: \Gamma \rightarrow \mathbb{R}\\$ 
		В силу спрямляемоси дуги $\exists l(\wideparen{P_k P_{k + 1}})$.
		Как всегда, рассматриваем разбиение $\tau: a = t_0 < t_1 < \dots < t_n = b$. 
		У нас появилось множество точек $P_k = (x(t_k), y(t_k), z(t_k))$.
		Пусть $\widetilde{P}_k = (x(\widetilde{t}_k), y(\widetilde{t}_k), z(\widetilde{t}_k))$, 
		где $\widetilde{t}_k \in [t_k, t_{k + 1}]$.
		Составляем интегральную сумму:
		\begin{gather}
			\sigma(\tau) = \sumlr{k = 0}{n - 1} f(\widetilde{P}_k)l(\wideparen{P_k P_{k + 1}})
		\end{gather}
		\begin{definition}
			$rang\tau = \max\limits_{k} l(\wideparen{P_k P_{k + 1}})$
		\end{definition}
		\begin{definition}[Криволинейный интеграл первого рода]
			Eсли $\exists \lim\limits_{rang\tau \rightarrow 0} \sigma(\tau)$, и он не зависит от 
			выбора промежуточных разбиений, то он называется криволинейным интегралом первого рода
		\end{definition}
	\item
		$f: \Gamma \rightarrow \mathbb{R}^3$.
		$\tau$ и $\widetilde{P}_k$ определяем так же.
		Пусть
		\begin{gather*}
		\Delta x_k = x(t_{k + 1}) - x(t_k),\:
		\Delta y_k = y(t_{k + 1}) - y(t_k),\:
		\Delta z_k = z(t_{k + 1}) - z(t_k).
		\end{gather*}
		Составим три интегральные суммы:
		\begin{gather}
			\sigma_x(\tau) = \sumlr{k = 0}{n - 1}f_x(\widetilde{P}_k)\Delta x_k \\
			\sigma_y(\tau) = \sumlr{k = 0}{n - 1}f_y(\widetilde{P}_k)\Delta y_k \\
			\sigma_z(\tau) = \sumlr{k = 0}{n - 1}f_z(\widetilde{P}_k)\Delta z_k
		\end{gather}
		\begin{definition}[Криволинейный интеграл второго рода]
			Если $\exists \lim\limits_{\text{rang}\tau \rightarrow 0}\sigma_x(\tau) = I_x, 
			\exists \lim\limits_{\text{rang}\tau \rightarrow 0}\sigma_y(\tau) = I_y,
			\exists \lim\limits_{\text{rang}\tau \rightarrow 0}\sigma_z(\tau) = I_z$
			Причем, они не зависят от выбора промежуточных разбиений, 
			то они называются криволинейными интегралами второго рода по координатным функциям
			и обозначаются:
			\begin{gather*}
				\intl{\Gamma}f_x(x, y, z)dx \defeq I_x \\
				\intl{\Gamma}f_y(x, y, z)dy \defeq I_y \\
				\intl{\Gamma}f_z(x, y, z)dz \defeq I_z 
			\end{gather*}
		\end{definition}
\end{enumerate}

\subsection{Вычисление криволинейных интегралов первого рода}

\begin{theorem}[О вычислении криволинейных интегралов первого рода]
	Если $f$ - непрерывна вдоль $\Gamma$, и $\Gamma$ - гладкая, 
	то криволинейный интеграл первого рода существует, и равен
	\begin{gather*}
		\intl{\Gamma}f dl = \intlr{a}{b} f(x(t), y(t), z(t))\sqrt{x'^2 + y'^2 + z'^2} dt
	\end{gather*}
	
	\begin{proof}
		Мы составляли интегральные суммы вида:
		\begin{gather*}
			\sigma(\tau) = \sumlr{k = 0}{n - 1} f(\widetilde{P}_k)l(\wideparen{P_k P_{k + 1}})
		\end{gather*}
		Так как $f(\widetilde{P}_k) = f(x(\widetilde{t}_k), y(\widetilde{t}_k), z(\widetilde{t}_k))$, 
		а $l(\wideparen{P_k P_{k + 1}}) = \intlr{t_k}{t_{k + 1}} \sqrt{x'^2 + y'^2 + z'^2}dt 
		\Rightarrow \frac{dl}{dt} = \sqrt{x'^2 + y'^2 + z'^2}$, причем последняя производная 
		одна и та же для всех $k$. Тогда
		$\intl{\Gamma} fdl = \intlr{t_0}{t_1} f(x(t), y(t), z(t))l' dt =\\
		\intlr{t_0}{t_1} f(x(t), y(t), z(t))\sqrt{x'^2 + y'^2 + z'^2} dt $
	\end{proof}
\end{theorem}

\subsection{Вычисление криволинейных интегралов второго рода}

\begin{theorem}[О вычислении криволинейных интегралов второго рода]
	Если $f$ - непрерывна вдоль $\Gamma$, и $\Gamma$ - гладкая, то 
	\begin{gather*}
		\intl{\Gamma} f_x dx + f_ydy + f_zdz = 
		\intlr{a}{b} f_x(x(t), y(t), z(t))x' dt + 
					f_y(x(t), y(t), z(t))y' dt +
					f_z(x(t), y(t), z(t))z' dt
	\end{gather*}
\end{theorem}

\begin{proof}
	для простоты докажем формулу
	$\intl{\Gamma} f_xdx = \intlr{a}{b}f_x(x(t), y(t), z(t))x' dt$. 
	Исходная получается аналогичным доказательством двух оставшихся, и применением свойства линейности. 
	Рассмотрим следующую интегральную сумму:
	\begin{gather*}
		\sigma(\tau) = \sumlr{k = 0}{n - 1} f(\overline{P}_k)x'(\overline{t}_k)\Delta t_k
		\xrightarrow[\text{rang}\tau \rightarrow 0]{} \intlr{a}{b} f x' dt
	\end{gather*}
	Теперь оценим модуль разности: 
	\begin{gather*}
		\abs{\sigma_x(\tau) - \sigma(\tau)} \leqslant 
		\sumlr{k = 0}{n - 1} \abs{f_x(\widetilde{P}_k) - f_x(\overline{P}_k)}
		\abs{x'(\overline{t}_k)} \Delta t_k
	\end{gather*}
	По теореме Кантора, $f_x(t)$ - равномерно непрерывна на $[a, b]$, это значит, что
	\begin{gather*}
		\feps > 0 \exists \delta : \text{rang}\tau \leqslant \delta \Rightarrow 
		\forall t', t'' \abs{f_x(t') - f_x(t'')} \leqslant \Epsilon  \: \Rightarrow\\
		\Rightarrow \abs{\sigma_x(\tau) - \sigma(\tau)} \leqslant 
		\Epsilon \sumlr{k = 0}{n - 1} x'(\overline{t}_k) \Delta t_k \leqslant \Epsilon M
	\end{gather*}
\end{proof}
\subsection{Формула Грина}

Существует связь между криволинейным интегралом второго рода по замкнутому контуру и 
двойным интегралом по внутренности этого контура. Она выражается в следующей теореме:

\begin{theorem}[Формула Грина]
	Пусть $P$ и $Q$ - непрерывно-дифференцируемы в односвязной области $G$. Пусть $\Gamma = \partial G$.
	Тогда:
	\begin{gather*}
		\intl{\Gamma_{+}} Pdx + Qdy = 
		\iintl{G} \left( \frac{\partial Q}{\partial x} - \frac{\partial P}{\partial y} \right) dxdy
	\end{gather*}
\end{theorem}

$\Gamma_{+}$ - означает, что обход такой, что внутренность $G$ всегда слева.

\begin{lemma}
	Пусть $G = \left\{(x, y) : a\leqslant x \leqslant b, f(x) \leqslant y \leqslant g(x)  \right\} $
	Пусть в $G \: \exists P$ - непрерывная и $\exists \frac{\partial P}{\partial y}$ - непрерывная.
	Тогда:
	\begin{gather*}
		\intl{\partial G_{+}} Pdx = -\iintl{G} \frac{\partial P}{\partial y} dxdy 
	\end{gather*}
\end{lemma}

\begin{proof}
	По теореме Фубини:
	\begin{gather*} 
		-\iintl{G} \frac{\partial P}{\partial y} dxdy = 
		-\intlr{a}{b}dx\intlr{f(x)}{g(x)} \frac{\partial P}{\partial y} dy = \\
		-\intlr{a}{b}\left( P(x, g(x)) - P(x, f(x)) \right) dx
	\end{gather*}
	\begin{figure}[h]
	\centering
	\begin{tikzpicture}
		\draw[->] (-0.1,0) -- (4.2,0) node[right] {$$};
		\draw[->] (0,-0.1) -- (0,4.2) node[above] {$$};
		\draw[scale=0.5,domain=1:7,smooth,variable=\x] node at (1.55,4) {\rom{4}}plot ({1},{\x});
		\draw[scale=0.5,domain=1:7,smooth,variable=\x] node at (7.5, 4) {\rom{2}} plot ({8},{\x});
		\draw (0.5,0.5) .. controls (1.66,1.5) and (2.83,-0.5) .. (4,0.5);
		\node at (2.25, 0.8) {\rom{1}};
		\node at (3.4,  3.7) {$g(x)$};
		\draw (0.5,3.5) .. controls (1.66,4.5) and (2.83,3) .. (4,3.5);
		\node at (2.25, 3.4) {\rom{3}};
		\node at (1.15, 0.5) {$f(x)$};
	\end{tikzpicture}
	\caption{множество $G$}
	\end{figure}
	Разделим интеграл на 4:
	\begin{gather*}
		\intl{\partial G_{+}} P dx = \intl{\text{\rom{1}}}Pdx + 
		\intl{\text{\rom{2}}}Pdx + \intl{\text{\rom{3}}}Pdx + \intl{\text{\rom{4}}}Pdx
	\end{gather*}
	\begin{gather*}
		\text{\rom{1}}: \begin{cases}
							x = t \\ 
							y = f(t)
						\end{cases} 
		\Rightarrow
		\intl{\text{\rom{1}}} P dx = \intlr{a}{b} P(t, f(t)) dt \\
		\text{\rom{3}}: \begin{cases}
							x = t \\ 
							y = g(t)
						\end{cases} 
		\Rightarrow
		\intl{\text{\rom{3}}_{-}} P dx = \intlr{a}{b} P(t, g(t)) dt 
	\end{gather*}
	На \rom{2} и \rom{4} $x = \text{const} \Rightarrow \intl{\text{\rom{2}}} Pdx = 
	\intl{\text{\rom{4}}}Pdx = 0$.
	Складывая, получаем, что правая часть формулы из условия, равна левой части
\end{proof}


\begin{nb}
	Если $G = \left\{ c \leqslant y \leqslant d, f(y) \leqslant x \leqslant g(y) \right\}$, то 
	\begin{gather*}
		\intl{\partial G_{+}} Qdy = \iintl{G} \frac{\partial Q}{\partial x} dxdy
	\end{gather*}
\end{nb}
\begin{proof}[Доказательство формулы Грина]
	Возьмем две точки на границе $G$. Соединим их жордановой кривой, 
	через внутренность $G$, тогда $G$ разделится на две области - $G_1$ и $G_2$, и
	\begin{gather*}
		\intl{\Gamma_{+}}Pdx + Qdy = \intl{\partial G_{1+}} Pdx + Qdy + 
		\intl{\partial G_{2+}} P dx + Qdy
	\end{gather*}
	Если мы научимся доказывать теорему для каких-то конкретных разделений $G$ на $G_1$ и
	$G_2$, то, теорема юудет доказана.
	Можно показать, что $G$ можно разбить на области удовлетворяющие лемме. 
	По аддитивности и линейности интеграла, в сумме они дают, исходную формулу.
\end{proof}
\newpage

\section{Поверхностные интегралы}

В этом параграфе рассматриваем двумерные поверхноси в трехмерном пространстве:
\begin{gather*}
	S:\begin{cases}
		x = x(u, v) \\
		y = y(u, v) , & (u, v) \in G \subset \mathbb{R}^2 \\
		z = z(u, v)
	  \end{cases}
\end{gather*}
Можно говорить о двусторонних и односторонних поверхностях. 
Для этого сначала введем понятние нормали к поверхности:
\begin{definition}[Нормаль к поверхности]
	Пусть 
	\large
	\begin{gather*}
		k_u = \left( \pdiff{x}{u}, \pdiff{y}{u}, \pdiff{z}{u}\right),
		k_v = \left( \pdiff{x}{v}, \pdiff{y}{v}, \pdiff{z}{v}\right) \\
		\text{Тогда вектор нормали } n = k_u \times k_v = 
		\begin{vmatrix}
			i & j & k \\ 
			\pdiff{x}{u} & \pdiff{y}{u} & \pdiff{z}{u} \\
			\pdiff{x}{v} & \pdiff{y}{v} & \pdiff{z}{v} \notag
		\end{vmatrix}
	\end{gather*}
\end{definition}

\begin{definition}[Двусторонняя поверхность]
	Поверхность называется двусторонней если ее вектор нормали непрерывен на ней.
\end{definition}

По определению, модуль векторного произведения, равен площади параллелограмма, построенного на множителях, это значит, что можно считать площадь поверхностей следующим образом:
\begin{gather*}
	\overline{N}_x = \frac{D(y, z)}{D(u, v)} = 
	\begin{vmatrix}
		\pdiff{y}{u} & \pdiff{y}{v} \\
		\pdiff{z}{u} & \pdiff{z}{v} \notag
	\end{vmatrix}; 
	\overline{N}_y = \frac{D(z, x)}{D(u, v)} = 
	\begin{vmatrix}
		\pdiff{z}{u} & \pdiff{z}{v} \\
		\pdiff{x}{u} & \pdiff{x}{v} \notag
	\end{vmatrix}; 
	\overline{N}_z = \frac{D(x, y)}{D(u, v)} = 
	\begin{vmatrix}
		\pdiff{x}{u} & \pdiff{x}{v} \\
		\pdiff{y}{u} & \pdiff{y}{v} \notag
	\end{vmatrix} \\
	\text{Тогда площадь бесконечно малого сегмента поверхности равна}
	\sqrt{\left( \frac{D(y,z)}{D(u,v)} \right)^2 + 
	\left( \frac{D(z, x)}{D(u,v)} \right)^2 +
	\left( \frac{D(x, y)}{D(u,v)}\right)^2} dudv
\end{gather*}
Тогда площадь поверхности
\begin{gather*}
	mes(S) \defeq\iintl{G} 
	\sqrt{\left( \frac{D(y,z)}{D(u,v)} \right)^2 + 
	\left( \frac{D(z, x)}{D(u,v)} \right)^2 +
	\left( \frac{D(x, y)}{D(u,v)}\right)^2} dudv
\end{gather*}
Дальше как всегда, делаем разбиение:
$\\ \tau = t_{11} < t_{12} < \dots < t_{1n}\\ t_{21} < t_{22} < \dots < t_{2n}\\ 
\dots\\ t_{n1} < t_{n2} < \dots < t_{nn}, t_{ij} \in G, 
t_{1j}, t_{i1}, t_{nj}, t_{in}\in \partial G$ и 
$\\ S_{ij} = 
\left\{ (x, y, z) : x = x(u,v), y = y(u,v), z = z(u,v), (u,v) \in G, 
t_{ij} \leqslant u \leqslant t_{i + 1 j}, t_{ij} \leqslant v \leqslant t_{i j + 1}\right\},
\\ P_{ij} \in S_{ij}$.
Теперь у нас все готово для определения поверхностного интеграла.

\subsection{Поверхностный интеграл первого рода}

Рассмотрим функционал $f(x, y, z), (x, y, z) \in S$.
Можно составить интегральную сумму 
\begin{gather*}
	\sigma(\tau) = \suml{ij} f(P_{ij}) mes(S_{ij})
	\xrightarrow[\text{rang}\tau \rightarrow 0]{} \iintl{S}f(x, y, z)dS
\end{gather*}
И тогда
\begin{gather*}
	\iintl{S}f(x,y,z)dS = \iintl{G}f(x(u,v), y(u,v), z(u, v))
	\sqrt{\left( \frac{D(y,z)}{D(u,v)} \right)^2 + 
	\left( \frac{D(z, x)}{D(u,v)} \right)^2 +
	\left( \frac{D(x, y)}{D(u,v)}\right)^2} dudv
\end{gather*}

\subsection{Поверхностный интеграл второго рода}

Рассмотрим функцию $\vec{f} : \mathbb{R}^3 \rightarrow \mathbb{R}^3$. Интегральные суммы определяются аналогично - умножаем координатную функцию на элемент площади. Тогда
\begin{gather*}
	\iintl{S} \vec{f}_xdydz + \vec{f}_ydzdx + \vec{f}_z dxdy \defeq 
	\iintl{S}\left( \vec{f}, \vec{n}\right)dS
\end{gather*}

Приведем некоторые определения из дифференциального исчисления:

\begin{definition}[Оператор Гамильтона (набла)]
	\begin{gather*}
		\nabla \defeq 
		\left( \pdiff{}{x}, \pdiff{}{y}, \pdiff{}{z}\right)
	\end{gather*}
\end{definition}

\begin{definition}[Градиент скалярного поля]
	\begin{gather*}
		\nabla f = \left( \pdiff{f}{x}, \pdiff{f}{y}, \pdiff{f}{z}\right)
		\defeq \grad f
	\end{gather*}
\end{definition}

\begin{definition}[Дивергенция векторного поля]
	\begin{gather*}
		\left( \nabla, \vec{F}\right) = 
		\pdiff{\vec{F}_x}{x} + \pdiff{\vec{F}_y}{y} + \pdiff{\vec{F}_z}{z} \defeq \Div \vec{F}
	\end{gather*}
\end{definition}

\begin{definition}[Ротор векторного поля]
	\begin{gather*}
		\nabla \times \vec{F} = 
		\begin{vmatrix}
			i & j & k \\
			\pdiff{}{x} & \pdiff{}{y} & \pdiff{}{z} \\
			\vec{F}_x & \vec{F}_y & \vec{F}_z \notag
		\end{vmatrix}
		\defeq \rot \vec{F}
	\end{gather*}
\end{definition}

\begin{definition}[Оператор Лапласа (Лапласиан)]
	\begin{gather*}
	\Delta \defeq 
		 \frac{\partial^2}{\partial x^2} +
		 \frac{\partial^2}{\partial y^2} + 
		 \frac{\partial^2}{\partial z^2}
	\end{gather*}
\end{definition}

\begin{theorem}[О треугольниках]
	\begin{gather*}
		\nabla^2 = \Delta
	\end{gather*}
\end{theorem}

\begin{proof}
	\begin{gather*}
		\nabla^2 = \left( \nabla , \nabla \right) = 
		\frac{\partial^2}{\partial x^2} +
		\frac{\partial^2}{\partial y^2} + 
		\frac{\partial^2}{\partial z^2}
		= \Delta
	\end{gather*}
\end{proof}

Следующая теорема позволяют сводить контурные интегралы второго рода к 
поверхностным интегралам второго рода.

\begin{theorem}[Теорема Стокса (без доказательства)]
	\begin{gather*}
		\oint\limits_{\partial S_{+}} \vec{F}_x dx + \vec{F}_y dy + \vec{F}_z dz =
		\iintl{S} \left( \rot \vec{F}, \vec{n}\right)dS
	\end{gather*}
\end{theorem}
Теорема Остроградского-Гаусса позволяет вычислять криволинейные интегралы 
по замкнутому контуру через поверхностные интегралы второго рода
\begin{theorem}[Теорема Остроградского-Гаусса (без доказательства)]
	Пусть $T = \partial S$
	\begin{gather*}
		\iintl{S} \left( \vec{F}, \vec{r}\right)dS = 
		\iiint\limits_{T} \Div \vec{F} dxdydz
	\end{gather*}
\end{theorem}

\begin{nb}
	Все эти теоремы являются частным случаем одной, доказаной Анри Картаном в 
	\rom{20} веке:
	\begin{gather*}
		\intl{\partial S} \omega = \intl{S} d\omega
	\end{gather*}
	Где $S$ - многообразие, $\omega$ - дифференицальная форма
\end{nb}


\end{document}
