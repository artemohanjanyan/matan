\documentclass{article}

\usepackage{amsmath}
\usepackage{amssymb}
\usepackage{amsthm}
\usepackage{mathtext}
\usepackage[T1,T2A]{fontenc}
\usepackage[utf8]{inputenc}
\usepackage[russian]{babel}
%\usepackage{geometry}
\usepackage[left=2cm,right=2cm,top=2cm,bottom=2cm,bindingoffset=0cm]{geometry}
\usepackage[mathscr]{euscript}
\usepackage{microtype}
\usepackage{bnf}
\usepackage{enumitem}
\usepackage{bm}
\usepackage{listings}
\usepackage{cancel}
\usepackage{proof}
\usepackage{epigraph}
\usepackage{titlesec}
\usepackage{mathtools}
%\setmainfont[Ligatures=TeX,SmallCapsFont={Times New Roman}]{Palatino Linotype}

\selectlanguage{russian}


\title{Математический анализ 4 семестр}
\author{shared with $\heartsuit$ by artemZholus}
\date{}

\begin{document}

\theoremstyle{definition}
\newtheorem*{definition}{Определение}
\theoremstyle{plain}
\newtheorem{theorem}{Теорема}[section]
\newtheorem{axiom}{Аксиома}
\newtheorem{lemma}[theorem]{Лемма}
\newtheorem{statement}[theorem]{Утверждение}
\newtheorem{corollary}[theorem]{Следствие}
\theoremstyle{remark}
\newtheorem*{example}{Пример}
\newtheorem{property}[theorem]{Свойство}


\newcommand{\todo}{\textsc{\textbf{TODO}}}
\newcommand{\abs}[1]{\left|#1\right|}
\newcommand{\intl}[1]{\int\limits_{#1}}
\newcommand{\intlr}[2]{\int\limits_{#1}^{#2}}
\newcommand{\suml}[1]{\sum\limits_{#1}}
\newcommand{\sumlr}[2]{\sum\limits_{#1}^{#2}}
\newcommand{\defeq}{\mathrel{\stackrel{\makebox[0pt]{\mbox{\normalfont\tiny def}}}{=}}}
\newcommand{\feps}{\forall\varepsilon}
\newcommand{\Epsilon}{\varepsilon}
\maketitle
\tableofcontents
%\newpage
\newcommand{\sectionbreak}{\clearpage}

\section{Критерий Лебега интегрируемости по Риману}

\begin{definition}[Колебание на отрезке]
	\[ \omega(f, c, d) = \sup\limits_{[c, d]} f - \inf\limits_{[c, d]} f =  
    \text{(по лемме из 1го семестра)}  = \sup\limits_{x',x'' \in [c,d]} | f(x') - f(x'') |\]
\end{definition}

\begin{definition}[Колебание функции в точке]
	\[ \omega(f, x) = \lim\limits_{\delta \rightarrow 0} \omega(f, x + \delta, x - \delta)\]
\end{definition}

Очевидно, колебание на отрезке неотрицательно, и, если $0 < \delta_1 < \delta_2$ 
то $\omega(f, x - \delta_1, x + \delta_1) < \omega(f, x - \delta_2, x + \delta_2)$.
Поэтому, вышеприведенный предел существует.

\begin{statement}
    $\omega(f, x) = 0 \Leftrightarrow f \in C(x)$
\end{statement}

\begin{proof}
	\begin{enumerate}
		\item $\Leftarrow$ Раз функция непрерывна, значит она достигает на отрезке своего $\sup$ и $\inf$. 
            Значит, если устремить границы отрезка к одной точке, в пределе получим разность двух одинаковых чисел.
		\item $\Rightarrow$ $\omega(f, x) = 0$ означает, что можно подобрать такую $\delta-$окрестность для $x$, 
            что она будет сколь угодно малой. Берем формулу $\sup\limits_{x',x'' \in [x - \delta,x + \delta]} | f(x') - f(x'') | = 0$ 
            фиксируем $x'' = x$ (от этого $\sup$ разве что уменьшится) и получаем определение непрерывности в $x$.
	\end{enumerate}
\end{proof}

\begin{definition}
$\tau:$ - разбиение отрезка $[a, b]$, если $\tau = \{x_j\}: \: a = x_0 < x_1 < \dots < x_n = b$
\end{definition}

Ведем кусочно-постоянную функцию $g(\tau, x) = \omega(f, x_j, x_{j + 1}),$ при $x \in [x_j, x_{j + 1}]$

\begin{statement}
$g(\tau_n, x) \xrightarrow[n \rightarrow +\infty]{} \omega(f, x)$ почти всюду на отрезке
\end{statement}

\begin{proof}
    Очевидно, мы можем подбирать $\tau_n$ так, чтобы границы отрезка, содержащего $x$ 
    совпали с границами из определения $\omega(f,x)$. Тогда для неграничных точек получим стремление. 
    Граничных точек на конечном шаге - конечное число, а это значит, что мы не перейдем за границу счетной 
    мощности (danger zone - МАТЛОГИКА), и предел будет почти всюду 
\end{proof}

Тогда, по теореме Лебега о предельном переходе под знаком интеграла, получаем:

\begin{gather*}
    \int\limits_{[a, b]}g(\tau_n, x)dx \rightarrow \int\limits_{[a,b]}\omega(f,x)dx
\end{gather*}

Левая часть, по лемме из первого семестра равна $\int\limits_{[a, b]}g(\tau_n, x)dx = \omega(f, \tau_n)$.
Получаем:

\begin{gather*}
    \lim\limits_{rang\tau_n \rightarrow 0} \omega(f, \tau_n) = \int\limits_{[a,b]} \omega(f,x)dx
\end{gather*}

Это наша рабочая формула.

\begin{theorem}[Критерий Лебега интегрируемости по Риману]
    $\\ f \in \Re (a, b) \Leftrightarrow \lambda\{ a : f \notin C(a) \} = 0$
\end{theorem}

\begin{proof}
	\begin{enumerate}
		\item 
			$\Rightarrow \\$ Пусть $\omega(f,x) = 0$ почти всюду на $[a, b]$. Тогда $\int\limits_{[a,b]} \omega(f, x)dx = 0 \: \Rightarrow f \in \Re [a,b]$ (Напрямую следует из утверждения $1.2$) 
		\item 
			$\Leftarrow \\$ Пусть $f \in \Re [a, b]$. Тогда, по определению, $\omega(f, \tau_n) \rightarrow 0$. 
            Тогда $\int\limits_{[a,b]} \omega(f, x)dx = 0$. Но $\omega(f, x) \geqslant 0$. 
            Значит $\omega(f,x) = 0$ почти всюду на $[a,b]$ (И, по лемме, почти всюду непрерывна).
	\end{enumerate}
\end{proof}




\section{Cуммируемые функции}
\subsection{Неотрицательные суммируемые функции}

Здесь и далее считаем, что мера $\mu$ - полная и $\sigma$-конечная.
Наша задача - распространить интеграл Лебега на более широкую ситуацию. Считаем, что $E \in \mathscr{A}$, $f: E \xrightarrow[]{измеримо}\mathbb{R}$, $f(x) \geqslant 0$ на $E$.

\begin{definition}
	$e \subset E$ называется допустимым для $f$ если:
	\begin{enumerate}
	\item
		$\mu(e) < +\infty$
	\item
		$f$ - ограничена на $e$
	\end{enumerate}
\end{definition}

\begin{statement}
	Непустые допустимые множества существуют.
\end{statement}

\begin{proof}
	Пусть $E_n = E(n < f(x) \leqslant n + 1)$. Понятно, что $E = \bigcup\limits_{n} E_n$. По $\sigma$-конечности $X = \bigcup\limits_{m}X_m$, причем $X_m$  - конечномерны. Тогда $E = \bigcup\limits_{n,m} E_nX_m$ - допустимые множества. Если они все пустые, то $E$, тоже пусто. Значит среди них хотя бы ожно непустое.
\end{proof}

\begin{definition}[Несобственный интеграл Лебега]
	\[\int\limits_{E} f d\mu \defeq \sup\limits_{e - допустимо} \int\limits_{e} fd\mu\]
\end{definition}

\begin{definition}[Неотрицательная суммируемая функция]
	Неотрицательная функция $f$ называется суммируемой на множестве $E$, если $\int\limits_{E} f d\mu < +\infty$
\end{definition}

Очевидно, если $\mu E < +\infty, \: f(x) \geqslant 0,$ то $\int\limits_{E} f d\mu = \sup\limits_{e \subset E} \int\limits_{e} fd\mu$.

Проверим аддитивность и линейность.

\begin{theorem}[$\sigma$-aддитивность несобственного интеграла Лебега]
	$\\*$Пусть $E = \bigcup\limits_{n} E_n$ - дизъюнктны. Тогда $\int\limits_{E} f = \sum\limits_{n} \int\limits_{E_n} f$
\end{theorem}

Докажем в два этапа. сначала конечную аддитивность, потом $\sigma$-аддитивность

\begin{proof}
	\begin{enumerate}
		\item 
			Пусть $E = E_1 \cup E_2$.Пусть $e_1 \in E_1$, $e_2 \in E_2$ - допустимые. И любое допустимое для $E$ множество $e = e_1 \cup e_2$.
			Для определенного интеграла мы знаем, что $\int\limits_{e} f = \int\limits_{e_1} f + \int\limits_{e_2} f \leqslant \int\limits_{E_1} f + \int\limits_{E_2} f$
			Переходя к $\sup$ по $e$ получаем $\int\limits_{E} f \leqslant \int\limits_{E_1} f + \int\limits_{E_2} f \\$
			В обратную сторону. Считаем, что $f$ - суммируема (иначе все тривиально). По определению $\sup$, $\feps > 0 \: \exists e_j \subset E_j : \int\limits_{E_j} f - \Epsilon < \int\limits_{e_j} f$. $\\$
			$\int\limits_{E_1} f + \int\limits_{E_1} f - 2\Epsilon < \int\limits_{e_1} f + \int\limits_{e_2} f = \int\limits_{e} f \leqslant \int\limits_{E} f $. Устремив $\Epsilon \rightarrow 0$ получим $\int\limits_{E_1} f + \int\limits_{E_2} f \leqslant \int\limits_{E} f. \\ $
			Значит $\int\limits_{E_1} f + \int\limits_{E_2} f  = \int\limits_{E} f $
		\item
			Итак, пусть $e = \bigcup\limits_{n = 1}^{+\infty} e_n$. Очевидно $\int\limits_{e_n} f \leqslant  \int\limits_{E_n} f$ и $ \int\limits_{e} f = \sum\limits_{n}\int\limits_{e_n} f$. Значит $ \int\limits_{E} f \leqslant  \sum\limits_{n} \int\limits_{E_n} f . \\$
			Обратно. $\feps > 0 \: \exists e_n \subset E_n : \\
			\int\limits_{E_n} f - \frac{\Epsilon}{2^n} <  \int\limits_{e_n} f $.  Сложим первые $p$ неравенств:$ \sum\limits_{1}^{p}\int\limits_{E_n} f - \Epsilon \sum\limits_{1}^{p} \frac{1}{2^n} < \sum\limits_{1}^{p}  \int\limits_{e_n} f \leqslant  \int\limits_{E} f$. Устремляя $p \rightarrow +\infty$, получаем $ \sum\limits_{n = 1}^{+\infty}\int\limits_{E_n} f - \Epsilon \leqslant  \int\limits_{E} f$. Теперь устремим $\Epsilon \rightarrow 0$ и получим обратное неравенство.
	\end{enumerate}
\end{proof}

\newpage

\begin{theorem}[Линейность несобственного интеграла Лебега]
	$\\*$
	\begin{enumerate}
		\item
			 $ \int\limits_{E} \alpha f  = \alpha  \int\limits_{E} f, \:\: \alpha > 0$
		\item
			 $ \int\limits_{E} (f + g) =  \int\limits_{E} f +  \int\limits_{E} g$
	\end{enumerate}
\end{theorem}

\begin{proof}
	Первое свойство следует непосредственно из определения. Докажем второе.
	Итак, пусть $E_n = E(n < f + g \leqslant n + 1)$. Тогда, очевидно, $E = \bigcup\limits_{n} E_n$. По $\sigma$-конечности можно написать $X = \bigcup\limits_{n} X_n$.
	От $X_n$ мы хотим дизъюнктности, поэтому, если они не таковы, то проделаем следующий трюк: $\\* X = X_1 \cup (X_2 \setminus X_1) \cup \dots \cup (X_n \setminus \bigcup\limits_{1}^{n-1}X_j) \cup \dots$.
	Теперь  $E$ можно разбить как $E = \bigcup\limits_{n, m} E_n X_m$ - эти множества дизъюнктны и допустимы для $f + g$. Далее по $\sigma$-аддитивности пишем: 
	$ \int\limits_{E} (f + g) = \sum\limits_{n}  \int\limits_{A_n} (f + g) = $ (по линейности определенного интеграла) $ =  \sum\limits_{n} \int\limits_{A_n} f + \sum\limits_{n} \int\limits_{A_n} g = $(по $\sigma$-аддитивности несобственного)$ = \int\limits_{E} f + \int\limits_{E} g$
\end{proof}

\begin{statement}
	Если $0 \leqslant f \leqslant g$, то $\int\limits_{E} f \leqslant \int\limits_{E} g$
\end{statement}

\begin{proof}
	$0 \leqslant g - f$ - по арифметике измеримости, эта функция суммируема. Раз она неотрицательна, интеграл от нее тоже. \[0 \leqslant \int\limits_{E} g - f =  \int\limits_{E} g -  \int\limits_{E} f \: \Rightarrow  \int\limits_{E} f \leqslant  \int\limits_{E} g\]
\end{proof}

\subsection{Суммируемые функции произвольного знака}

\begin{definition}
	\[
	f^+(x) = 
	\begin{cases}
		0 &, f(x) < 0 \\
		f(x) &, f(x) \geqslant 0
	\end{cases}
	\]
	$\\*$
	\[
	f^-(x) = 
	\begin{cases}
		-f(x) &, f(x) < 0 \\
		0 &, f(x) \geqslant 0
	\end{cases}
	\]
\end{definition}

Заметим, что $f = f^+ - f^-$, $|f| = f^+ + f^-$. $f^+$ и $f^-$ - неотрицательные суммируемые функции (если $f$ - измерима).


\begin{definition}
	$f$ называется суммируемой на $E$, если одновременно $f^+$ и $f^-$ - суммируемы.
	\[ \int\limits_{E} f \defeq  \int\limits_{E} f^+ -  \int\limits_{E} f^- \]
\end{definition}

\begin{statement}
	$f$ - суммируема $\Leftrightarrow$ $|f|$ - суммируема.
\end{statement}

\begin{proof}
	$f$ - суммируема тогда и только тогда когда $f^+$ и $f^-$ - суммируемы. $|f|$ - суммируема тогда и только тогда, когда $f^+$ и $f^-$ - суммируемы. 
\end{proof}

Проверим $\sigma$-аддитивность и линейность для случая функции произвольного знака:

\newpage

\begin{theorem}[Аддитивность в сдучае произвольного знака]
	Пусть $E = \bigcup\limits_{n} E_n$ - дизъюнктные, тогда $\int\limits_{E} f = \sum\limits_{n} \int\limits_{E_n} f$
\end{theorem}

\begin{proof}
	$\int\limits_{E} f^+ = \sum\limits_{n} \int\limits_{E_n} f^+$, то же для $f^-$. Тогда $\int\limits_{E} f = \int\limits_{E} f^+ +  \int\limits_{E} f^- = \sum\limits_{n}\int\limits_{E_n} f^+ + \sum\limits_{n}\int\limits_{E_n} f^- = \sum\limits_{n} (\int\limits_{E_n} f^+ - f^-) = \sum\limits_{n}\int\limits_{E_n} f $
\end{proof}


\begin{theorem}[Линейность в случае произвольного знака]
        $\\*$
        \begin{enumerate}
                \item
			$ \int\limits_{E} \alpha f  = \alpha  \int\limits_{E} f, \: \alpha \in \mathbb{R}$
                \item
                         $ \int\limits_{E} (f + g) =  \int\limits_{E} f +  \int\limits_{E} g$
        \end{enumerate}
\end{theorem}

\begin{proof}
	Пункт 1 очевиден, не будем на нем останавливаться. Докажем пункт 2:$\\*$
	\[
		\int\limits_{E} f + \int\limits_{E} g = (\int\limits_{E} f^+ \int\limits_{E} g^+) - (\int\limits_{E} f^- + \int\limits_{E} g^-) = \int\limits_{E} (f^+ + g^-) - \int\limits_{E} (f^- + g^-) = (\ast) = \int\limits_{E} (f + g)^+ - \int\limits_{E} (f + g)^- = \int\limits_{E} (f + g)
	\]
	Проверим переход  $(\ast)$. Для этого нужно, чтобы выполнялось $(f^+ + g^+) = (f + g)^+$ - в общем случае, это неправда. Поэтому нужно рассмотреть много случаев:
	\begin{enumerate}
		\item
			$f \geqslant 0, \: g\geqslant 0 \Rightarrow$ пусть $E_1 = E(f \geqslant 0, \: g \geqslant 0)$
		\item
			$ f \leqslant 0, \: g \leqslant 0 \Rightarrow$ пусть $E_2 = E(f \leqslant 0, \: g \leqslant 0)$
		\item
			$f \geqslant 0, \: g\leqslant 0 \Rightarrow$ тут нужно различить два подслучая:
			\begin{enumerate}
				\item 
					$f + g \geqslant 0 \Rightarrow$ пусть $E_3 = E(f \geqslant 0, \: g\leqslant 0, f + g \geqslant 0)$
				\item
					$f + g < 0 \Rightarrow$ пусть $E_4 = E(f \geqslant 0, \: g\leqslant 0, f + g < 0)$
			\end{enumerate}
		\item
			$f \leqslant 0, \: g\geqslant 0 \Rightarrow$ аналогично, два подслучая:
			\begin{enumerate}
				\item 
					$f + g \geqslant 0 \Rightarrow$ пусть $E_5 = E(f \leqslant 0, \: g\geqslant 0, f + g \geqslant 0)$
				\item
					$f + g < 0 \Rightarrow$ пусть $E_6 = E(f \leqslant 0, \: g\geqslant 0, f + g < 0)$
			\end{enumerate}
	\end{enumerate}
	Очевидно, эти множества дизъюнктны (на $0$ забьем) и можно написать: $\int\limits_{E} f = \sum\limits_{1}^{6}\int\limits_{E_j} f$.
	Дальше идет нудный разбор случаев, я потом напишу \todo
\end{proof}



\section{Предельный переход в классе суммируемых функций}

\subsection{Теорема Лебега о мажорируемой сходимости}

\begin{theorem}[Теорема Лебега о мажорируемой сходимости]
	Пусть $f_n \Rightarrow f$ на $E$, $|f_n| \leqslant \phi$ на $E$, $\phi$ - суммируема.
	$\\$ Тогда:
	\begin{enumerate}
		\item
			$f$ - суммируема
		\item
			$\intl{E} f_n \rightarrow \intl{E} f$
	\end{enumerate}
\end{theorem}

\begin{center}
	\framebox[4in]{
		\begin{minipage}[t]{3.5in}% я дезигнер
			Следует иметь ввиду, что в условии теоремы достаточно требовать выполнения свойств почти всюду.
		\end{minipage}
	}	
\end{center}

\begin{theorem}
	Пусть $f$ - суммируема на $E$. Тогда $\feps > 0 \: \exists \delta > 0 : \forall E' \subset E \Rightarrow \mu E' < \delta \Rightarrow |\intl{E'} f| < \Epsilon$ 
\end{theorem}

\begin{proof}
	По определению, можно написать $\feps > 0 \: \exists e : \intl{E \setminus e} |f| < \Epsilon$. Так как $e$ - допустимо, $f$ - ограничена на $e$ и $E = (E \setminus e) \cup e$. Возьмем любое $E' \subset E$, тогда $E' = E'(E \setminus e) \cup E'e$.
	\[
		\abs{\intl{E'} f} \leqslant \abs{\intl{E'(E \setminus e)} f} + \abs{\intl{E'e} f} \leqslant \Epsilon + \abs{\intl{E'e} f}
	\]
	Мы считаем, что $\abs{f(x)} \leqslant M$. Заметим, что выбор $E'$ не накладывал никаких ограничений на $M$. Тогда:
	\[
		\intl{E'e} \abs{f} \leqslant M \mu E'e \leqslant M \mu E'
	\]
	Поэтому $\delta$ мы можем выбрать как $\delta = \frac{\Epsilon}{M}$. И получится, что $\mu E' \leqslant \delta \Rightarrow \abs{\intl{E'} f} \leqslant 2\Epsilon$
\end{proof}

\begin{proof}[Доказательство теоремы Лебега]
	По теореме Рисса $f_{n_k} \rightarrow f$ почти всюду, причем $\abs{f_{n_k}(x)} \leqslant \phi(x)$, занчит $\abs{f(x)} \leqslant \phi(x) \: \Rightarrow f -$ суммируема. 	Рассмотрим $\abs{\intl{E} f_n  - \intl{E} f} \leqslant \intl{E} \abs{f_n - f}$. Так как $\phi $ - суммируема, $\\* \feps > 0 \:\: \exists e \text{(допустимое для $\phi$)} : \intl{E \setminus e} \phi \leqslant \Epsilon$ 
	\[
		\intl{E} \abs{f_n - f} = \intl{E \setminus e} \abs{f_n - f} + \intl{e} \abs{f_n - f} \leqslant 2\Epsilon + \intl{e} \abs{f_n - f}
	\]
	Пусть $\abs{\phi} \leqslant M \Rightarrow \abs{f_n - f} \leqslant 2M$. Так же мы знаем, что $\intl{e} \abs{f_n - f} \xrightarrow[n \rightarrow +\infty]{} 0$. Значит, начиная с некоторого $N_0$, $\intl{e} \abs{f_n - f} < \Epsilon$.
	Следовательно, начиная с $N_0$, $\intl{E} \abs{f_n - f} \leqslant 3\Epsilon$
\end{proof}

\newpage

\subsection{Теорема Леви}

\begin{theorem}[Теорема Леви]
	Пусть $f_n(x) \leqslant 0$, $f_n(x) \leqslant f_{n + 1}(x)$, $f(x) = \lim\limits_{n\rightarrow +\infty} f_n(x)$ на $E$. 
	Тогда $\intl{E} f_n \rightarrow \intl{E} f$
\end{theorem}


\begin{proof}
	Два случая:
	\begin{enumerate}
		\item
			$f$ - почти всюду конечна на $E$.
			Два подслучая:
			\begin{enumerate}
				\item
					$\intl{E} f < +\infty$. Так как $\abs{f_n(x)} \leqslant f(x) \: \Rightarrow$ $f$ - суммируемая мажоранта для $f_n$, и теорема верна по теореме Лебега о мажорируемой сходимости.
				\item
					$\intl{E} f = +\infty$. ($f$ все еще мажоранта для $f_n$, но уже не суммируемая) Мы поступим так. Раз $\sup\limits_{e - допустимо} \intl{e} f = +\infty$, значит $\forall c > 0 \: \exists e - допустимое для f : c < \intl{e} f$. В силу $f_n \leqslant f$  по теореме Лебега о мажорируемой сходимости $\intl{e} f_n \rightarrow \intl{e} f$. Это значит, начиная с некоторого $N_0$, $c < \intl{e} f_n \leqslant \intl{E} f_n \: \Rightarrow \intl{E} f_n \rightarrow +\infty = \intl{E} f$
			\end{enumerate}
		\item
			$\mu E(f = +\infty) > 0$ (Расслабьтесь, и будет не больно)$\\$ 
			Очевидно, в этой ситуации может быть только $\intl{E} f = +\infty$. Из $f_n(x) \leqslant f_{n + 1}(x) \: 
			\Rightarrow \intl{E} f_{n}(x) \leqslant \intl{E} f_{n + 1}(x)$. 
			По теореме Вейерштрасса, у последовательности $\left\{ \intl{E} f_n \right\}$ будет существовать предел. 
			Причем он будет конечным тогда и только тогда, когда эта последовательность ограничена. 
			Так что нам нужно вывести противоречие из того факта, что эта последовательность ограничена.
			Предположим, что это так: пусть $\intl{E} f_n \leqslant M$. Итак, зафиксируем $\forall c > 0$. Рассмотрим $E(f_n \geqslant c) \subset E$.
			\begin{gather*}
				\intl{E(f_n \geqslant c)} f_n \leqslant M \\ c \mu E(f_n \geqslant c) \leqslant \intl{E(f_n \geqslant c)} f_n \Rightarrow 
				\mu E(f_n \geqslant c) \leqslant \frac{M}{c}
			\end{gather*}
			Можно проверить, что:
			\begin{gather*}
				E(f = +\infty) \subset \bigcup\limits_{m = 1}^{\infty} \bigcap\limits_{n = m}^{\infty} E(f_n \geqslant c)
			\end{gather*}
			\begin{proof}
				Пусть $x \in E(f = +\infty)$. Значит $f_n(x) \xrightarrow[n \rightarrow +\infty]{} +\infty$. 
				Следовательно $\forall c > 0 \exists N_x : \forall n > N_x \Rightarrow f_n(x) \geqslant c \xRightarrow[def]{by} 
				x \in \bigcap\limits_{n = N_x}^{\infty} E(f_n \geqslant c)$
			\end{proof}
			Заметим одну интересную штуку. 
			\begin{gather*}
				\forall c > 0 f_n(x) \geqslant c \Rightarrow f_{n + 1}(x) \geqslant c \Rightarrow 
				E(f_n \geqslant c) \subset E(f_{n + 1} \geqslant c) \Rightarrow \bigcap\limits_{n = m}^{\infty} E(f_n \geqslant c) = E(f_m \geqslant c)
				\Rightarrow \\
				\Rightarrow \bigcup\limits_{m = 1}^{\infty} \bigcap\limits_{n = m}^{\infty} E(f_n \geqslant c) = 
				\lim\limits_{m \rightarrow +\infty} E(f_m \geqslant c)
			\end{gather*}
			Отсюда делаем вывод, что: 
			\begin{gather*}
				\mu \bigcap\limits_{n = m}^{\infty} E(f_n \geqslant c) \xrightarrow[m \rightarrow +\infty]{} 
				\mu \bigcup\limits_{m = 1}^{\infty} \bigcap\limits_{n = m}^{\infty} E(f_n \geqslant c) \geqslant
				\mu E(f = +\infty)
			\end{gather*}
			\newpage
			\begin{gather*}
				\mu \bigcap\limits_{n = m}^{\infty} E(f_n \geqslant c) = \mu E(f_m \geqslant c) \leqslant \frac{M}{c} \Rightarrow \\
				\Rightarrow \mu \bigcup\limits_{m = 1}^{\infty} \bigcap\limits_{n = m}^{\infty} E(f_n \geqslant c) \leqslant \frac{M}{c} \Rightarrow \\
				\Rightarrow \mu E(f = +\infty) \leqslant \frac{M}{c}
			\end{gather*}
			$c$ - любое, поэтому можно устремить $c \rightarrow +\infty$. Значит $\mu E(f = +\infty) = 0$. Противоречие получено.
	\end{enumerate}
\end{proof}

\begin{corollary}
	Пусть $u_n(x) \geqslant 0$ и $\suml{n}\intl{E}u_n$ - сходится. Тогда $\suml{n}u_n(x)$ - сходится почти всюду на $E$.
\end{corollary}

\begin{proof}
	$S_n = u_1 + u_2 + \dots + u_n$. Так как интеграл сходится, его частичная сумма ограничена. $M \geqslant \sumlr{n = 1}{m}\intl{E} u_n = \intl{E} S_m$		
	Обозначим $S(x) = \suml{n} u_n(x)$. В силу неотрицательности $u_n(x)$, $S_n(x)$ - возрастает ($S_n \leqslant S_{n + 1}$), и 
	$S(x) = \lim\limits_{n \rightarrow +\infty} S_n(x)$. Следовательно, по теореме Леви $\intl{E} S_n \rightarrow \intl{E} S$. Следовательно, $S$ - суммируемая функция, это значит, что она почти всюду конечна. 
\end{proof}

\begin{corollary}
	Пусть $f \geqslant 0,  \:\: f_n(x) = \min\left\{f(x), n\right\}$ - срезка функции $f$. Тогда $\intl{E} f_n \rightarrow \intl{E} f$.
\end{corollary}

\begin{proof}
	$f_n$ удовлетворяют условиям теоремы Леви.
\end{proof}

\subsection{Теорема Фату}

\begin{theorem}[Теорема Фату]
	Пусть $f_n \geqslant 0$, $f_n \Rightarrow f$ на $E$. Тогда \[\intl{E} f \leqslant \sup\limits_{n \in \mathbb{N}} \intl{E} f_n	\]
\end{theorem}

\begin{proof}
	Применим теорему Рисса, получив, что $f_{n_k} \rightarrow f$ почти всюду. Без ограничения общности можем считать, что $f_n \rightarrow f$ почти всюду
	(потому что если доказать для $\sup$ по подпоследовательности, неравенство будет верно и для последовательности).
	Пусть $g_n = \min\left\{f_n, f\right\}$. Тогда $g_n \leqslant f$.
	Рассмотрим два случая:
	\begin{enumerate}
		\item 
			$f$ - суммируема.
			Тогда, по теореме Лебега, $\intl{E} g_n \rightarrow \intl{E} f$. Предел последовательности $\intl{E} g_n$ не превзойдет своего верхнего предела, 
			поэтому $\intl{E} f \leqslant \sup\limits_{n} \intl{E} g_n \leqslant \sup\limits_{n}\intl{E} f_n$
		\item
			$\intl{E} f = +\infty$. Тогда $\forall e$ - допустимо для $f$. $\intl{e} f < +\infty$
			Как мы показали, $\intl{e} f \leqslant \sup\limits_{n}\intl{E} f_n$. Переходя к $\sup$ по $e$ получаем необходимое неравенство.
	\end{enumerate}
\end{proof}

\newpage

\section{Пространства $L_p$}

\begin{definition}
	$L_p(E) \defeq \left\{f : E \rightarrow \mathbb{R}, f - \text{измерима}, \intl{E} \abs{f}^p < +\infty \right\}$
\end{definition}

Нам нужно проверить, что $L_p(E)$ - НП. То есть, $f,g \in L_p \Rightarrow \alpha f + \beta g \in L_p$, $\norm{f}_p = \left( \intl{E} \abs{f}^p \right)^{\frac{1}{p}}$.
Причем, $\norm{f}$ - удовлетворяет аксиомам нормы. 

\begin{statement}
	$\normp{f}$ удовлетворяет двум свойствам:
	\begin{enumerate}
		\item
			$\normp{\alpha f } = |\alpha| \normp{f}$
		\item
			$\normp{f + g} \leqslant \normp{f} + \normp{g}$
	\end{enumerate}
\end{statement}

\begin{proof}
	\begin{enumerate}
		\item
			Очевидно.
		\item
			$\intl{E} \abs{f + g}^p \leqslant \intl{E} (\abs{f} + \abs{g})^p$. Пусть $E_1 = E(\abs{f} \leqslant \abs{g}), \: E_2 = E(\abs{f} > \abs{g})$.
			Тогда $E = E_1 \cup E_2$. 
			\begin{gather*}
				\intl{E} (\abs{f} + \abs{g})^p = \intl{E_1} (\abs{f} + \abs{g})^p + \intl{E_2} (\abs{f} + \abs{g})^p \leqslant \\
				\leqslant \intl{E_1} (2\abs{g})^p + \intl{E_2} (2\abs{f})^p < +\infty
			\end{gather*}
			Следовательно $f + g \in L_p$
	\end{enumerate}
\end{proof}

\begin{theorem}[Неравенство Гельдера]
	Пусть $p > 1$ и $q : \frac{1}{p} + \frac{1}{q} = 1$. Пусть $f \in L_p, \: g \in L_q$. Тогда
	\begin{gather*}
		\intl{E} \abs{f}\abs{g} \leqslant \left( \intl{E} \abs{f}^p \right)^{\frac{1}{p}}  \left( \intl{E} \abs{g}^q \right)^{\frac{1}{q}}
	\end{gather*}
\end{theorem}


\begin{proof}
	Воспользуемся неравенством Юнга: ($uv \leqslant \frac{1}{p}u^p + \frac{1}{q}v^q$).
	Пусть $u =  \frac{\abs{f}}{\normp{f}}$, $v = \frac{\abs{g}}{\normp{g}}$
	\begin{gather*}
		\frac{\abs{f}\abs{g}}{\normp{f}\normp{g}} \leqslant \frac{1}{p} \frac{\abs{f}^p}{\normp{f}^p} + \frac{1}{q} \frac{\abs{g}^q}{\normp{g}^q}\\
		\intl{E}\frac{\abs{f}\abs{g}}{\normp{f}\normp{g}} \leqslant 
		\frac{1}{p} \intl{E}\frac{\abs{f}^p}{\normp{f}^p} + 
		\frac{1}{q} \intl{E}\frac{\abs{g}^q}{\normp{g}^q} = \frac{1}{p} + \frac{1}{q} = 1 
	\end{gather*}
\end{proof}

\begin{theorem}[Неравенство Минковского]
	Пусть $p > 1$, $f, g \in L_p$. Тогда
	\begin{gather*}
		\left(\intl{E} (\abs{f} + \abs{g})^p \right)^{\frac{1}{p}} \leqslant 
		\left( \intl{E} \abs{f}^p \right)^{\frac{1}{p}} + 
		\left( \intl{E} \abs{g}^p \right)^{\frac{1}{p}}
	\end{gather*}
\end{theorem}

\begin{proof}
	Рассмотрим $(f + g)^p = f(f + g)^{p - 1} + g(f + g)^{p - 1}$.
	\begin{gather*}
		\intl{E} (f + g)^p = \intl{E} f(f + g)^{p - 1} + \intl{E}g(f + g)^{p - 1} \leqslant \\
		\leqslant \left( \intl{E} f^p \right)^{\frac{1}{p}} \left(\intl{E} (f + g)^{q(p - 1)} \right)^{\frac{1}{q}} + 
	  		  \left( \intl{E} g^p \right)^{\frac{1}{p}} \left(\intl{E} (f + g)^{q(p - 1)} \right)^{\frac{1}{q}} \\
		\text{пусть}\: q = \frac{p}{p - 1} \\
		\left(\intl{E}(f + g)^p \right)^{1 - \frac{1}{q}} \leqslant  \left( \intl{E} f^p \right)^{\frac{1}{p}} +  \left( \intl{E} g^p \right)^{\frac{1}{p}} \\
		\frac{1}{p} = 1 - \frac{1}{q}
	\end{gather*}
\end{proof}

Если подставить в неравенство Минковского определение нормы, то можно заметить, что мы доказали неравенство треугольника.

\begin{theorem}
	$L_p(E) $ - Банахово пространство.
\end{theorem}

Докажем вспомогательную лемму:

\begin{lemma}
	Пусть $f_n$ - измеримы, и $\forall\delta > 0\: \mu E\left(\abs{f_n - f_m} \geqslant \delta\right) \xrightarrow[n,m \rightarrow +\infty]{} 0$.
	Тогда $\exists n_1 < n_2 < \dots < n_k < \dots : f_{n_k} \rightarrow f$ почти всюду.
\end{lemma}

\begin{proof}
	Пусть $\delta = \frac{1}{2^k}$. Можно проверить, что (\todo) 
	$\exists n_1 < n_2 < \dots < n_k < \dots: \\$
	$\mu E\left(\abs{f_{n_{k + 1}} - f_{n_k}} \geqslant \frac{1}{2^k}\right) \leqslant \frac{1}{2^k}$.
	Рассмотрим следующее множество:
	\begin{gather*}
		E' = \bigcup\limits_{k = 1}^{\infty}\bigcap\limits_{j = k}^{\infty} E\left(\abs{f_{n_{j + 1}} - f_{n_j}} \leqslant \frac{1}{2^j} \right)
	\end{gather*}
	Рассмотрим функциональный ряд $S = f_1 + (f_2 - f_1) + (f_3 - f_2) + \dots$. Фиксируем $x \in E'$. Тогда 
	$\\ \exists k_x : x \in \bigcap\limits_{j = k_x}^{\infty} E\left(\abs{f_{n_{j + 1}} - f_{n_j}} \leqslant \frac{1}{2^j} \right)$.
	Это значит, что при $j > k_x$ выполняется $\abs{f_{n_{j + 1}}(x) - f_{n_k}(x)} \leqslant \frac{1}{2^j} \xrightarrow[j \rightarrow +\infty]{} 0$.
	Следовательно, на $E'$ функциональный ряд $S$ - сходится. 
	Нам осталось проверить, что его дополнение нуль-мерно.
	Т. е. $\mu \overline{E'} = 0$. Очевидно:
	\begin{gather*}
		\overline{E'} = \bigcap\limits_{k = 1}^{\infty}\bigcup\limits_{j = k}^{\infty} E\left(\abs{f_{n_{j + 1}} - f_{n_j}} > \frac{1}{2^j} \right) \Rightarrow \\
		\Rightarrow \overline{E'} \subset \bigcup\limits_{j = k}^{\infty} E\left(\abs{f_{n_{j + 1}} - f_{n_j}} > \frac{1}{2^j} \right) \Rightarrow \\
		\Rightarrow \mu \overline{E'} \leqslant \sumlr{j = k}{\infty} \mu E(\abs{f_{n_{j + 1}} - f_{n_j}} > \frac{1}{2^j}) \leqslant 
		\sumlr{j = k}{\infty} \frac{1}{2^j} \xrightarrow[k \rightarrow +\infty]{} 0 \Rightarrow \\
		\Rightarrow \mu \overline{E'} = 0
	\end{gather*}
\end{proof}

\newpage

\begin{proof}[Доказательство Теоремы]
\end{proof}

Может показаться, что требование измеримости функции $f$ в определении пространства $L_p$ -  излишне. Это отнюдь не так.
\begin{statement}
	Существует функция $f$ такая, что ее $p$-я степень измерима, но сама функция - нет. 
\end{statement}

\begin{proof}
	Рассмотрим произвольное неизмеримое множество $C \subset \mathbb{R}$. Тогда пусть
	\begin{gather*}
		f(x) = 
		\begin{cases}
			1  &, x \in C \\
			-1 &, x \notin C
		\end{cases}
	\end{gather*}
	Очевидно, $f$ -неизмерима (так как множество Лебега $E(f > \frac{1}{2}) = С$ - неизмеримо). Но $f^2(x) = 1$ при $x \in \mathbb{R}$ - очевидно, измеримая функция.
\end{proof}


\end{document}
