
\section{Cуммируемые функции}
\subsection{Неотрицательные суммируемые функции}

Здесь и далее считаем, что мера $\mu$ - полная и $\sigma$-конечная.
Наша задача - распространить интеграл Лебега на более широкую ситуацию. Считаем, что $E \in \mathscr{A}$, $f: E \xrightarrow[]{измеримо}\mathbb{R}$, $f(x) \geqslant 0$ на $E$.

\begin{definition}
	$e \subset E$ называется допустимым для $f$ если:
	\begin{enumerate}
	\item
		$\mu(e) < +\infty$
	\item
		$f$ - ограничена на $e$
	\end{enumerate}
\end{definition}

\begin{statement}
	Непустые допустимые множества существуют.
\end{statement}

\begin{proof}
	Пусть $E_n = E(n < f(x) \leqslant n + 1)$. Понятно, что $E = \bigcup\limits_{n} E_n$. По $\sigma$-конечности $X = \bigcup\limits_{m}X_m$, причем $X_m$  - конечномерны. Тогда $E = \bigcup\limits_{n,m} E_nX_m$ - допустимые множества. Если они все пустые, то $E$, тоже пусто. Значит среди них хотя бы ожно непустое.
\end{proof}

\begin{definition}[Несобственный интеграл Лебега]
	\[\int\limits_{E} f d\mu \defeq \sup\limits_{e - допустимо} \int\limits_{e} fd\mu\]
\end{definition}

\begin{definition}[Неотрицательная суммируемая функция]
	Неотрицательная функция $f$ называется суммируемой на множестве $E$, если $\int\limits_{E} f d\mu < +\infty$
\end{definition}

Очевидно, если $\mu E < +\infty, \: f(x) \geqslant 0,$ то $\int\limits_{E} f d\mu = \sup\limits_{e \subset E} \int\limits_{e} fd\mu$.

Проверим аддитивность и линейность.

\begin{theorem}[$\sigma$-aддитивность несобственного интеграла Лебега]
	$\\*$Пусть $E = \bigcup\limits_{n} E_n$ - дизъюнктны. Тогда $\int\limits_{E} f = \sum\limits_{n} \int\limits_{E_n} f$
\end{theorem}

Докажем в два этапа. сначала конечную аддитивность, потом $\sigma$-аддитивность

\begin{proof}
	\begin{enumerate}
		\item 
			Пусть $E = E_1 \cup E_2$.Пусть $e_1 \in E_1$, $e_2 \in E_2$ - допустимые. И любое допустимое для $E$ множество $e = e_1 \cup e_2$.
			Для определенного интеграла мы знаем, что $\int\limits_{e} f = \int\limits_{e_1} f + \int\limits_{e_2} f \leqslant \int\limits_{E_1} f + \int\limits_{E_2} f$
			Переходя к $\sup$ по $e$ получаем $\int\limits_{E} f \leqslant \int\limits_{E_1} f + \int\limits_{E_2} f \\$
			В обратную сторону. Считаем, что $f$ - суммируема (иначе все тривиально). По определению $\sup$, $\feps > 0 \: \exists e_j \subset E_j : \int\limits_{E_j} f - \Epsilon < \int\limits_{e_j} f$. $\\$
			$\int\limits_{E_1} f + \int\limits_{E_1} f - 2\Epsilon < \int\limits_{e_1} f + \int\limits_{e_2} f = \int\limits_{e} f \leqslant \int\limits_{E} f $. Устремив $\Epsilon \rightarrow 0$ получим $\int\limits_{E_1} f + \int\limits_{E_2} f \leqslant \int\limits_{E} f. \\ $
			Значит $\int\limits_{E_1} f + \int\limits_{E_2} f  = \int\limits_{E} f $
		\item
			Итак, пусть $e = \bigcup\limits_{n = 1}^{+\infty} e_n$. Очевидно $\int\limits_{e_n} f \leqslant  \int\limits_{E_n} f$ и $ \int\limits_{e} f = \sum\limits_{n}\int\limits_{e_n} f$. Значит $ \int\limits_{E} f \leqslant  \sum\limits_{n} \int\limits_{E_n} f . \\$
			Обратно. $\feps > 0 \: \exists e_n \subset E_n : \\
			\int\limits_{E_n} f - \frac{\Epsilon}{2^n} <  \int\limits_{e_n} f $.  Сложим первые $p$ неравенств:$ \sum\limits_{1}^{p}\int\limits_{E_n} f - \Epsilon \sum\limits_{1}^{p} \frac{1}{2^n} < \sum\limits_{1}^{p}  \int\limits_{e_n} f \leqslant  \int\limits_{E} f$. Устремляя $p \rightarrow +\infty$, получаем $ \sum\limits_{n = 1}^{+\infty}\int\limits_{E_n} f - \Epsilon \leqslant  \int\limits_{E} f$. Теперь устремим $\Epsilon \rightarrow 0$ и получим обратное неравенство.
	\end{enumerate}
\end{proof}

\newpage

\begin{theorem}[Линейность несобственного интеграла Лебега]
	$\\*$
	\begin{enumerate}
		\item
			 $ \int\limits_{E} \alpha f  = \alpha  \int\limits_{E} f, \:\: \alpha > 0$
		\item
			 $ \int\limits_{E} (f + g) =  \int\limits_{E} f +  \int\limits_{E} g$
	\end{enumerate}
\end{theorem}

\begin{proof}
	Первое свойство следует непосредственно из определения. Докажем второе.
	Итак, пусть $E_n = E(n < f + g \leqslant n + 1)$. Тогда, очевидно, $E = \bigcup\limits_{n} E_n$. По $\sigma$-конечности можно написать $X = \bigcup\limits_{n} X_n$.
	От $X_n$ мы хотим дизъюнктности, поэтому, если они не таковы, то проделаем следующий трюк: $\\* X = X_1 \cup (X_2 \setminus X_1) \cup \dots \cup (X_n \setminus \bigcup\limits_{1}^{n-1}X_j) \cup \dots$.
	Теперь  $E$ можно разбить как $E = \bigcup\limits_{n, m} E_n X_m$ - эти множества дизъюнктны и допустимы для $f + g$. Далее по $\sigma$-аддитивности пишем: 
	$ \int\limits_{E} (f + g) = \sum\limits_{n}  \int\limits_{A_n} (f + g) = $ (по линейности определенного интеграла) $ =  \sum\limits_{n} \int\limits_{A_n} f + \sum\limits_{n} \int\limits_{A_n} g = $(по $\sigma$-аддитивности несобственного)$ = \int\limits_{E} f + \int\limits_{E} g$
\end{proof}

\begin{statement}
	Если $0 \leqslant f \leqslant g$, то $\int\limits_{E} f \leqslant \int\limits_{E} g$
\end{statement}

\begin{proof}
	$0 \leqslant g - f$ - по арифметике измеримости, эта функция суммируема. Раз она неотрицательна, интеграл от нее тоже. \[0 \leqslant \int\limits_{E} g - f =  \int\limits_{E} g -  \int\limits_{E} f \: \Rightarrow  \int\limits_{E} f \leqslant  \int\limits_{E} g\]
\end{proof}

\subsection{Суммируемые функции произвольного знака}

\begin{definition}
	\[
	f^+(x) = 
	\begin{cases}
		0 &, f(x) < 0 \\
		f(x) &, f(x) \geqslant 0
	\end{cases}
	\]
	$\\*$
	\[
	f^-(x) = 
	\begin{cases}
		-f(x) &, f(x) < 0 \\
		0 &, f(x) \geqslant 0
	\end{cases}
	\]
\end{definition}

Заметим, что $f = f^+ - f^-$, $|f| = f^+ + f^-$. $f^+$ и $f^-$ - неотрицательные суммируемые функции (если $f$ - измерима).


\begin{definition}
	$f$ называется суммируемой на $E$, если одновременно $f^+$ и $f^-$ - суммируемы.
	\[ \int\limits_{E} f \defeq  \int\limits_{E} f^+ -  \int\limits_{E} f^- \]
\end{definition}

\begin{statement}
	$f$ - суммируема $\Leftrightarrow$ $|f|$ - суммируема.
\end{statement}

\begin{proof}
	$f$ - суммируема тогда и только тогда когда $f^+$ и $f^-$ - суммируемы. $|f|$ - суммируема тогда и только тогда, когда $f^+$ и $f^-$ - суммируемы. 
\end{proof}

Проверим $\sigma$-аддитивность и линейность для случая функции произвольного знака:

\newpage

\begin{theorem}[Аддитивность в сдучае произвольного знака]
	Пусть $E = \bigcup\limits_{n} E_n$ - дизъюнктные, тогда $\int\limits_{E} f = \sum\limits_{n} \int\limits_{E_n} f$
\end{theorem}

\begin{proof}
	$\int\limits_{E} f^+ = \sum\limits_{n} \int\limits_{E_n} f^+$, то же для $f^-$. Тогда $\int\limits_{E} f = \int\limits_{E} f^+ +  \int\limits_{E} f^- = \sum\limits_{n}\int\limits_{E_n} f^+ + \sum\limits_{n}\int\limits_{E_n} f^- = \sum\limits_{n} (\int\limits_{E_n} f^+ - f^-) = \sum\limits_{n}\int\limits_{E_n} f $
\end{proof}


\begin{theorem}[Линейность в случае произвольного знака]
        $\\*$
        \begin{enumerate}
                \item
			$ \int\limits_{E} \alpha f  = \alpha  \int\limits_{E} f, \: \alpha \in \mathbb{R}$
                \item
                         $ \int\limits_{E} (f + g) =  \int\limits_{E} f +  \int\limits_{E} g$
        \end{enumerate}
\end{theorem}

\begin{proof}
	Пункт 1 очевиден, не будем на нем останавливаться. Докажем пункт 2:$\\*$
	\[
		\int\limits_{E} f + \int\limits_{E} g = (\int\limits_{E} f^+ \int\limits_{E} g^+) - (\int\limits_{E} f^- + \int\limits_{E} g^-) = \int\limits_{E} (f^+ + g^-) - \int\limits_{E} (f^- + g^-) = (\ast) = \int\limits_{E} (f + g)^+ - \int\limits_{E} (f + g)^- = \int\limits_{E} (f + g)
	\]
	Проверим переход  $(\ast)$. Для этого нужно, чтобы выполнялось $(f^+ + g^+) = (f + g)^+$ - в общем случае, это неправда. Поэтому нужно рассмотреть много случаев:
	\begin{enumerate}
		\item
			$f \geqslant 0, \: g\geqslant 0 \Rightarrow$ пусть $E_1 = E(f \geqslant 0, \: g \geqslant 0)$
		\item
			$ f \leqslant 0, \: g \leqslant 0 \Rightarrow$ пусть $E_2 = E(f \leqslant 0, \: g \leqslant 0)$
		\item
			$f \geqslant 0, \: g\leqslant 0 \Rightarrow$ тут нужно различить два подслучая:
			\begin{enumerate}
				\item 
					$f + g \geqslant 0 \Rightarrow$ пусть $E_3 = E(f \geqslant 0, \: g\leqslant 0, f + g \geqslant 0)$
				\item
					$f + g < 0 \Rightarrow$ пусть $E_4 = E(f \geqslant 0, \: g\leqslant 0, f + g < 0)$
			\end{enumerate}
		\item
			$f \leqslant 0, \: g\geqslant 0 \Rightarrow$ аналогично, два подслучая:
			\begin{enumerate}
				\item 
					$f + g \geqslant 0 \Rightarrow$ пусть $E_5 = E(f \leqslant 0, \: g\geqslant 0, f + g \geqslant 0)$
				\item
					$f + g < 0 \Rightarrow$ пусть $E_6 = E(f \leqslant 0, \: g\geqslant 0, f + g < 0)$
			\end{enumerate}
	\end{enumerate}
	Очевидно, эти множества дизъюнктны (на $0$ забьем) и можно написать: $\int\limits_{E} f = \sum\limits_{1}^{6}\int\limits_{E_j} f$.
	Дальше идет нудный разбор случаев, я потом напишу \todo
\end{proof}


