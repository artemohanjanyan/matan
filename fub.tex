\section{Теорема Фубини}

\begin{definition}[Линейное сечение множества $E$]
    $E_x \defeq \set{y \in \mathbb{R} : (x, y) \in E}$
\end{definition}

\begin{example}
    Пусть $E = [a, b] \times [c, d]$. Тогда 
    \begin{gather*}
        E_x = 
        \left\{ \begin{matrix}
                \varnothing, x \notin [a,b] 
                \\ 
                [c, d], x \in [a, b]
        \end{matrix}\right.
    \end{gather*}
\end{example}

\begin{theorem}
    Пусть $\lambda E < +\infty$, тогда
    \begin{enumerate}
        \item 
            Любое $E_x$ - измеримо.
        \item
            $\lambda E_x$ - измеримая, почти всюду конечная функция. 
        \item
            $\lambda E = \intl{\mathbb{R}} \lambda E_x dx$
    \end{enumerate}
\end{theorem}

\begin{proof}
    Пусть для начала $E = G$ - открытое множество. Тогда $G = \bigcup\limits_{m} \Pi_m$ - дизъюнктные ячейки. 
    Очевидно, $G_x = \bigcup\limits_{m} \Pi_{n, x}$. По $\sigma$. По $\sigma$ - аддитивности, $\lambda G_x = \suml{m}\lambda \Pi_{m, x}$.
    Каждое слагаемое, как функция, измеримо, значит, очевидно, будет измерима и сумма. 
    Поэтому, по теореме Леви, данное равенство можно интегрировать. 
    \begin{gather*}
        \intl{\mathbb{R}} \lambda G_x = \suml{m} \intl{\mathbb{R}} \lambda \Pi_{m ,x} = \suml{m} \lambda \Pi_m = \lambda G
    \end{gather*}

    Далее, пусть $E$ "--- произвольное измеримое множество. 
	 Воспользуемся формулой $\lambda E_x = \inf\limits_{E_x \subset G_x} \lambda G_x$, где $G_x$ "--- открытые. 
	 По аксиоме выбора, $\exists G_{x,m} : G_{x,m+1} \subset G_{x,m}, E_x = \bigcap\limits_{m} G_{x,m}$.
	 $\sigma$-алгебра замкнута относително счетных пересечений, поэтому $E_x$ "--- измеримо.  
	 Также понятно, что $\lambda G_{x,m} \rightarrow \lambda E_x$. Следовательно, по теореме о поточечном пределе измеримых функций $\lambda E_x$ "--- измерима
	 и почти всюду конечна. По теореме Лебеге о связи сходимости почти всюду и по мере, получаем сходимость помере, 
	 и пользуясь теоремой Лебега о предельном переходе под знаком интеграла получаем, что $\lambda E = \intl{\mathbb{R}} \lambda E_x$.
	 \begin{nb}
		 В условиии теоремы Лебега о предельном переходе требуется конечномерность множества, по которому мы интегрируем, 
		 это можно обойти воспользовавшись $\sigma$-конечностью меры "--- разобьем $\mathbb{R}$ в объединение конечномерных множеств,
		 для каждого из них воспользуемся теоремой Лебега, а затем по $\sigma$-аддитивности интеграла сложим их в один.
	 \end{nb}
\end{proof}

\begin{definition}
    Пусть $E \subset \mathbb{R}^2, f$ - измерима и неотрицательна. Тогда
    $\intl{E} f(x, y) d\lambda \defeq \iintl{E} f(x, y) dxdy$
\end{definition}

Теперь мы можем сформулировать теорему Фубини:

\begin{theorem}[Теорема Фубини]
    Пусть $E \subset \mathbb{R}^2, f$ - суммируема на $E$, тогда почти всюду $f(x, \cdot)$ - 
    суммируема на $E_x$, $\intl{E_x} f(x, y) dy$ -  суммируема на $\mathbb{R}$ и
    \begin{gather*}
        \iintl{E} f(x, y) dxdy = \intl{\mathbb{R}} dx \intl{E_x} f(x, y) dy
    \end{gather*}
\end{theorem}

\begin{proof}
    Достаточно доказать для неотрицательных функций. Для функций произвольного знака это будет 
    следовать из линейности двойного интеграла. 
    Итак, мы знаем, что подграфик $\\ G_f = \set{(x,y) \in \mathbb{R} : (x, y) \in E, 0 \leqslant z \leqslant f(x, y)}$ 
    измерим. $\iintl{E} f(x, y) dxdy = \lambda G_f$.
    \begin{gather*}
        G_{f,x} = \set{(y,z): (x, y)\in E, 0 \leqslant z \leqslant f(x,y)} =\\
        = \set{(y, z) : y \in E_x, 0 \leqslant z \leqslant f(x, y)} = G_{f(x,\cdot)} \Rightarrow
        \\ \Rightarrow \text{ по теореме о мере подграфика } \Rightarrow \lambda G_{f,x} = \intl{E_x} f(x, y)dy
    \end{gather*}
    Подставляя это в исходную формулу, получаем формулу повторного интегрирования.
\end{proof}
