\section{Ряды фурье}


Далее, если не оговорено обратного, все интегралы будут интегралами Лебега.
\begin{definition}[Тригонометриеский ряд]
	Тригонометрическим рядом называется сумма вида:
	\begin{gather*}
		\frac{a_0}{2} + \sumlr{n = 1}{\infty} a_n \cos(nx) + b_n \sin(nx)
	\end{gather*}
	Частичные суммы тригонометрического ряда обозначаются $T_n(x)$
\end{definition}

Докажем несколько вспомогательных утверждений:
\begin{statement}
	Если $m \neq  n$ то
\begin{gather*}
	\intl{Q}\cos(nx)\sin(mx)dx = \intl{Q}\cos(nx)\cos(mx)dx = \intl{Q}\sin(nx)\sin(mx) = 0
\end{gather*}
\end{statement}
	
\begin{proof}
	\begin{enumerate}
		\item 
			\begin{gather*}
				\intl{Q}\cos(nx)\sin(mx)dx = 
				\frac{1}{2}\intl{Q}\left( \sin(x(n + m)) - \sin(x(n - m))\right) dx =\\
				\frac{1}{2}\left( -\frac{1}{n + m}\cos(x(n + m))\right)\Big|_{-\pi}^{\pi} -
				\frac{1}{2}\left( \frac{1}{n - m}\sin(x(n - m))\right)\Big|_{-\pi}^{\pi} = 0
			\end{gather*}
		\item
			\begin{gather*}
				\intl{Q}\cos(nx)\cos(mx)dx = 
				\frac{1}{2}\intl{Q}\left( \cos(x(n - m)) + \cos(x(n + m))\right) dx =\\
				\frac{1}{2}\left( \frac{1}{n - m}\sin(x(n - m))\right)\Big|_{-\pi}^{\pi} +
				\frac{1}{2}\left( \frac{1}{n + m}\sin(x(n + m))\right)\Big|_{-\pi}^{\pi} = 0
			\end{gather*}
		\item
			\begin{gather*}
				\intl{Q}\sin(nx)\sin(mx)dx = 
				\frac{1}{2}\intl{Q}\left( \cos(x(n - m)) - \cos(x(n + m))\right) dx =\\
				\frac{1}{2}\left( \frac{1}{n - m}\sin(x(n - m))\right)\Big|_{-\pi}^{\pi} -
				\frac{1}{2}\left( \frac{1}{n + m}\sin(x(n + m))\right)\Big|_{-\pi}^{\pi} = 0
			\end{gather*}		
	\end{enumerate}
\end{proof}

\begin{theorem}
	Пусть тригонометрический ряд сходится в пространстве $L_1$.
	Тогда для его коэффициентов выполнены формулы Эйлера-Фурье:
	\begin{gather*}
		a_0 = \frac{1}{\pi}\intl{Q}f(x)dx \\
		a_n = \frac{1}{\pi}\intl{Q}f(x)cos(nx)dx \\
		b_n = \frac{1}{\pi}\intl{Q}f(x)sin(nx)dx
	\end{gather*}
\end{theorem}

\begin{proof}
	Пусть $m, n \in \mathbb{N}, n > m$.
	Оценим следующий интеграл:
	\begin{gather*}
		\abs{\intl{Q}\left( f(x)\cos(mx) - T_n(x)\cos(mx)\right)dx} = \\
		\abs{\intl{Q}f(x)\cos(mx)dx - \intl{Q}\sumlr{k = 0}{n}A_k(x)\cos(mx)dx}\\
		\intl{Q}\sumlr{k = 0}{n}A_k(x)\cos(mx)dx = \sumlr{k = 0}{n}\intl{Q}A_k(x)\cos(mx) dx =\\
		\frac{a_0}{2}\intl{Q}\cos(mx)dx + \sumlr{k = 0}{n}\left(a_k\intl{Q}\cos(kx)\cos(mx)dx +
		b_k\intl{Q}\sin(kx)\cos(mx)dx\right) = a_m \intl{Q}\cos^2(mx)dx = \pi a_m \\
		\Rightarrow = \abs{a_m \pi - \intl{Q} f(x) \cos(mx) dx}
	\end{gather*}
	С другой стороны, нельзя не согласиться, что 
	\begin{gather*}
		\abs{\intl{Q}\left( f(x)\cos(mx) - T_n(x)\cos(mx)\right)dx} \leqslant \\
		\intl{Q}\abs{f(x) - T_n(x)} \abs{\cos(mx)} \leqslant \intl{Q}\abs{f(x) - T_n(x)} 
		\xrightarrow[n \rightarrow +\infty]{} 0 \Rightarrow \\
		\abs{a_m \pi - \intl{Q} f(x) \cos(mx) dx} = 0
	\end{gather*}
\end{proof}

\begin{corollary}
	Если тригонометрический ряд равномерно сходится на $Q$, 
	то его коэффициенты вычисляются по формулам ЭЙлера-Фурье.
\end{corollary}

\begin{definition}
	Пусть $f \in L_1$, тогда тригонометрический ряд с коэффициентами, 
	вычисленными по формулам Эйлера-Фурье, называется рядом Фурье функции $f$. 
\end{definition}

\begin{nb}
	Существуют сходящиеся тригонометрические ряды, которые не являются рядами Фурье своих сумм.
\end{nb}

\newpage

\section{Интегралы Дирихле и Фейера}

\subsection{Интеграл Дирихле}

Итак, для вывода интеграла Дирихле, рассмотрим частичную сумму ряда Фурье:

\begin{gather*}
	T_n(x) = \intl{Q}f(t)\frac{dt}{2\pi} + 
	\sumlr{k = 1}{n} \left( \intl{Q} f(t)\frac{1}{\pi}\cos(kt)dt\right) \cos(kx) + 
					 \left(	\intl{Q} f(t)\frac{1}{\pi}\sin(kt)dt\right) \sin(kx) = \\
	\intl{Q}f(t)\frac{1}{\pi} 
	\left(\frac{1}{2} + \sumlr{k = 1}{n} \cos(kt)\cos(kx) + \sin(kt)\sin(kx) \right)dt = \\
	\intl{Q}f(t) \frac{1}{\pi} \left( \frac{1}{2} + \sumlr{k = 1}{n} \cos(k(t - x))\right)dt
\end{gather*}

\begin{definition}[Ядро Дирихле]
	\begin{gather*}
		D_n(t) \defeq \frac{1}{\pi} \left( \frac{1}{2} + \sumlr{k = 1}{n} \cos(kt)\right)
	\end{gather*}
\end{definition}


