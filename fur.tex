\section{Ряды фурье}

\subsection{Определение ряда Фурье}

Далее, если не оговорено обратного, все интегралы будут интегралами Лебега.
\begin{definition}[Тригонометриеский ряд]
	Тригонометрическим рядом называется сумма вида:
	\begin{gather*}
		\frac{a_0}{2} + \sumlr{n = 1}{\infty} a_n \cos(nx) + b_n \sin(nx)
	\end{gather*}
	Частичные суммы тригонометрического ряда обозначаются $T_n(x)$
\end{definition}

Докажем несколько вспомогательных утверждений:
\begin{statement}
	Если $m \neq  n$ то
\begin{gather*}
	\intl{Q}\cos(nx)\sin(mx)dx = \intl{Q}\cos(nx)\cos(mx)dx = \intl{Q}\sin(nx)\sin(mx) = 0
\end{gather*}
\end{statement}
	
\begin{proof}
	\begin{enumerate}
		\item 
			\begin{gather*}
				\intl{Q}\cos(nx)\sin(mx)dx = 
				\frac{1}{2}\intl{Q}\left( \sin(x(n + m)) - \sin(x(n - m))\right) dx =\\
				\frac{1}{2}\left( -\frac{1}{n + m}\cos(x(n + m))\right)\Big|_{-\pi}^{\pi} -
				\frac{1}{2}\left( \frac{1}{n - m}\sin(x(n - m))\right)\Big|_{-\pi}^{\pi} = 0
			\end{gather*}
		\item
			\begin{gather*}
				\intl{Q}\cos(nx)\cos(mx)dx = 
				\frac{1}{2}\intl{Q}\left( \cos(x(n - m)) + \cos(x(n + m))\right) dx =\\
				\frac{1}{2}\left( \frac{1}{n - m}\sin(x(n - m))\right)\Big|_{-\pi}^{\pi} +
				\frac{1}{2}\left( \frac{1}{n + m}\sin(x(n + m))\right)\Big|_{-\pi}^{\pi} = 0
			\end{gather*}
		\item
			\begin{gather*}
				\intl{Q}\sin(nx)\sin(mx)dx = 
				\frac{1}{2}\intl{Q}\left( \cos(x(n - m)) - \cos(x(n + m))\right) dx =\\
				\frac{1}{2}\left( \frac{1}{n - m}\sin(x(n - m))\right)\Big|_{-\pi}^{\pi} -
				\frac{1}{2}\left( \frac{1}{n + m}\sin(x(n + m))\right)\Big|_{-\pi}^{\pi} = 0
			\end{gather*}		
	\end{enumerate}
\end{proof}

\begin{theorem}
	Пусть тригонометрический ряд сходится в пространстве $L_1$.
	Тогда для его коэффициентов выполнены формулы Эйлера-Фурье:
	\begin{gather*}
		a_0 = \frac{1}{\pi}\intl{Q}f(x)dx \\
		a_n = \frac{1}{\pi}\intl{Q}f(x)cos(nx)dx \\
		b_n = \frac{1}{\pi}\intl{Q}f(x)sin(nx)dx
	\end{gather*}
\end{theorem}

\begin{proof}
	Пусть $m, n \in \mathbb{N}, n > m$.
	Оценим следующий интеграл:
	\begin{gather*}
		\abs{\intl{Q}\left( f(x)\cos(mx) - T_n(x)\cos(mx)\right)dx} = \\
		\abs{\intl{Q}f(x)\cos(mx)dx - \intl{Q}\sumlr{k = 0}{n}A_k(x)\cos(mx)dx}\\
		\intl{Q}\sumlr{k = 0}{n}A_k(x)\cos(mx)dx = \sumlr{k = 0}{n}\intl{Q}A_k(x)\cos(mx) dx =\\
		\frac{a_0}{2}\intl{Q}\cos(mx)dx + \sumlr{k = 0}{n}\left(a_k\intl{Q}\cos(kx)\cos(mx)dx +
		b_k\intl{Q}\sin(kx)\cos(mx)dx\right) = a_m \intl{Q}\cos^2(mx)dx = \pi a_m \\
		\Rightarrow = \abs{a_m \pi - \intl{Q} f(x) \cos(mx) dx}
	\end{gather*}
	С другой стороны, нельзя не согласиться, что 
	\begin{gather*}
		\abs{\intl{Q}\left( f(x)\cos(mx) - T_n(x)\cos(mx)\right)dx} \leqslant \\
		\intl{Q}\abs{f(x) - T_n(x)} \abs{\cos(mx)} \leqslant \intl{Q}\abs{f(x) - T_n(x)} 
		\xrightarrow[n \rightarrow +\infty]{} 0 \Rightarrow \\
		\abs{a_m \pi - \intl{Q} f(x) \cos(mx) dx} = 0
	\end{gather*}
\end{proof}

\begin{corollary}
	Если тригонометрический ряд равномерно сходится на $Q$, 
	то его коэффициенты вычисляются по формулам ЭЙлера-Фурье.
\end{corollary}

\begin{definition}
	Пусть $f \in L_1$, тогда тригонометрический ряд с коэффициентами, 
	вычисленными по формулам Эйлера-Фурье, называется рядом Фурье функции $f$. 
\end{definition}

\begin{nb}
	Существуют сходящиеся тригонометрические ряды, которые не являются рядами Фурье своих сумм.
\end{nb}

\newpage

\subsection{Интеграл Дирихле}

Итак, для вывода интеграла Дирихле, рассмотрим частичную сумму ряда Фурье:

\begin{gather*}
	T_n(x) = \intl{Q}f(t)\frac{dt}{2\pi} + 
	\sumlr{k = 1}{n} \left( \intl{Q} f(t)\frac{1}{\pi}\cos(kt)dt\right) \cos(kx) + 
					 \left(	\intl{Q} f(t)\frac{1}{\pi}\sin(kt)dt\right) \sin(kx) = \\
	\intl{Q}f(t)\frac{1}{\pi} 
	\left(\frac{1}{2} + \sumlr{k = 1}{n} \cos(kt)\cos(kx) + \sin(kt)\sin(kx) \right)dt = \\
	\intl{Q}f(t) \frac{1}{\pi} \left( \frac{1}{2} + \sumlr{k = 1}{n} \cos(k(t - x))\right)dt
\end{gather*}

\begin{definition}[Ядро Дирихле]
	\begin{gather*}
		D_n(t) \defeq \frac{1}{\pi} \left( \frac{1}{2} + \sumlr{k = 1}{n} \cos(kt)\right)
	\end{gather*}
\end{definition}

\begin{definition}[Интеграл Дирихле]
	Частичные суммы тригонометрического ряда в следующем виде называют интегралом Дирихле.
	\begin{gather*}
		T_n(x) = \intl{Q} f(t) D_n(x - t)dt
	\end{gather*}
\end{definition}

Так как $D_n(t) = D_n(-t) \Rightarrow 
T_n(x) = \intlr{0}{\pi} \left( f(x + t ) + f(x -t )\right)D_n(t)dt$ 
(Здесь мы пользуемся периодичностью $f$).

\begin{statement}
	\begin{gather*}
		D_n(t) = \frac{1}{2\pi}\frac{\sin((n + \frac{1}{2})t)}{\sin\frac{t}{2}}
	\end{gather*}
\end{statement}

\begin{proof}
	\begin{gather*}
		\sin\frac{t}{2}D_n(t) =
		\frac{1}{\pi} \left( \frac{1}{2}\sin\frac{t}{2} +
		\sumlr{k = 1}{n} \cos(kt)\sin\frac{t}{2}\right) = 
		\frac{1}{\pi} \left( \frac{1}{2}\sin\frac{t}{2}+ 
		\frac{1}{2}\sumlr{k = 1}{n} \sin(k + \frac{1}{2})t - \sin(k - \frac{1}{2})t\right) = \\
		\frac{1}{2\pi}\sin(n + \frac{1}{2})t
	\end{gather*}
\end{proof}

Из формулы ясно, что $D_n(t)$ меняет знак с увеличением  $n$ все чаще, и можно показать, что 
\begin{gather*}
	\intl{Q} \abs{D_n(t) dt} \sim \ln n
\end{gather*}

Это заметно усложняет изучение рядов фурье в отдельных точках.

Очевидно, $\intl{Q}D_n(t) dt = 1$, следовательно $f(x) = \intlr{0}{\pi} 2 f(x) D_n(t) dt$.
Тогда можно записать разность в следующем виде:
\begin{gather*}
	T_n(x) - f(x) = \intlr{0}{\pi} \left(f(x + t) + f(x - t) - 2 f(x)\right)D_n(t) dt
\end{gather*}

Обозначим
$\phi(x, t) \defeq \left(f(x + t) + f(x - t) - 2 f(x)\right)$

И тогда получится 
\begin{gather*}
	T_n(x) - f(x) = \intlr{0}{\pi} \phi(x, t) D_n(t) dt
\end{gather*}

Эта формула нам пригодится в дальнейшем.

\subsection{Лемма Римана-Лебега}

\begin{lemma}[Лемма Римана-Лебега]
	Пусть $f \in L_1$ - суммируема на $\mathbb{R}$. Тогда
	\begin{gather*}
		\intl{\mathbb{R}} f(x) \cos nx dx \xrightarrow[n \rightarrow +\infty]{} 0\\
		\intl{\mathbb{R}}f(x) \sin nx dx  \xrightarrow[n \rightarrow +\infty]{} 0
	\end{gather*}
\end{lemma}

\begin{corollary}
	Если $f \in L_1$, то $a_n, b_n \xrightarrow[n \rightarrow +\infty]{} 0$
\end{corollary}

\begin{proof}[Доказательство следствия]
	$\pi a_n(f) = \intl{Q} f(x)\cos nx dx$. Пусть
	\begin{gather*}
		g(x) = 
		\left\{\begin{matrix}
			f(x), & x\in Q\\
			0,   & \text{otherwise} 
		\end{matrix}\right.
	\end{gather*}
	Тогда $\intl{\mathbb{R}}\abs{g} = \intl{Q}\abs{f} <+\infty$. 
	Следовательно $g$ - суммируема на числовой оси. 
	Следовательно $\intl{\mathbb{R}}g \cos nx dx \xrightarrow[n \rightarrow +\infty]{}0$.
	Но $\intl{\mathbb{R}}\abs{g} = \intl{Q}\abs{f} = \pi a_n(f)$
\end{proof}

\begin{proof}[Доказательство Леммы]
	Для доказательства леммы воспользуемся теоремой Лузина. 
	Для забывчивых, напомним ее формулировку:
	\begin{theorem}[Теорема Лузина]
		Пусть $E \subset \mathbb{R}^n$, $f \text{ - измерима на } E$. Тогда 
		$\feps > 0 \exists\phi \text{ - непрерывная на } \mathbb{R}^n : 
		\mu(f \neq \phi) < \Epsilon$. И если $\abs{f(x)}\leqslant M$ на $E$, то 
		$\abs{\phi(x)} \leqslant M$ на $\mathbb{R}$.
	\end{theorem}
	Доказательство будем вести в несколько этапов.
	Сначала покажем, что для любой функции суммируемой на оси и $\feps > 0 \exists g$ - 
	непрерывная и ограниченная на оси, такая, что $\normpp{f - g}{1} \leqslant \Epsilon$. 

	Так как $\intl{\mathbb{R}}f < +\infty$, то, по свойствам интеграла, 
	$\feps > 0$ существует множество конечной меры $E$ на котором функция $f$ - ограничена, 
	и при этом
	$\intl{\mathbb{R}\setminus E}\abs{f} < \Epsilon$. Теперь, по теореме Лузина,
	подберем $g$ - непрерывную на $\mathbb{R}$ и ограниченую $M$ функцию.
	Определим 
	\begin{gather*}
		\hat{f}(x) = 
		\left\{\begin{matrix}
			f(x), & x \in E \\
			0, & \text{otherwise}
		\end{matrix}\right.
	\end{gather*}
	Тогда $\intl{\mathbb{R}} \abs{f - \hat{f}} dx = \intl{\mathbb{R}\setminus E} \abs{f} < \Epsilon$.
	Ясно, что, так как $f$ - ограничена на $E$, 
	то $\hat{f}$ - ограничена той же константой на всей оси.
	По теореме Лузина, $\lambda\mathbb{R}(f \neq g) < \frac{\Epsilon}{M}$; 
	обозначим $E' = \mathbb{R}(f \neq g)$.
	Теперь 
	\begin{gather*}
		\intl{\mathbb{R}}\abs{f - g} \leqslant \intl{\mathbb{R}}\abs{f - \hat{f}} +
		\intl{\mathbb{R}}\abs{\hat{f} - g}
	\end{gather*}
	\begin{enumerate}
		\item $\intl{\mathbb{R}} \abs{\hat{f} - f} < \Epsilon$
		\item $\intl{\mathbb{R}} \abs{\hat{f} - g} = \intl{E'} \abs{\hat{f} - g} \leqslant 
			2M \lambda E' = 2M \frac{\Epsilon}{M} = 2\Epsilon$
	\end{enumerate}
	\begin{gather*}
		\intl{\mathbb{R}} \abs{f - g} \leqslant 3\Epsilon
	\end{gather*}
	
	Терперь проверим лемму Римана-Лебега для $g$.
	Мера Лебега $\sigma$-конечна, из этого следует, что(это надо чекнуть \todo)
	в качестве $E$ можно брать конечный промежуток.
	$E = \left< a, b\right>$. Рассмотрим $\abs{\intl{\mathbb{R}} g(x) \cos nx dx} = 
	\intl{\mathbb{R} \setminus \left< a, b \right>} \abs{g(x) \cos nx }dx +
	\intl{\left<a, b \right>} \abs{g(x) \cos nx} dx$
	\begin{enumerate}
		\item $\intl{\mathbb{R} \setminus \left< a, b \right>} \abs{g(x) \cos nx }dx \leqslant 
			\intl{\mathbb{R} \setminus \left< a, b \right>} \abs{g(x)}dx  < \Epsilon$
		\item $\abs{\intl{\left<a, b \right>} g(x) \cos nx} dx = 
			\abs{\intlr{a}{b} g(x)d\left( \frac{1}{n}\sin nx \right)} \leqslant 
			\abs{g(x)\frac{1}{n}\sin nx\Big|_{a}^{b}} + 
			\abs{\frac{1}{n}\intlr{a}{b}\sin nx d\left( g(x)\right)}$
			\begin{enumerate}
				\item $g(x)\frac{1}{n}\sin nx\Big|_{a}^{b} \xrightarrow[n \rightarrow +\infty]{} 0$
				\item $\abs{\frac{1}{n}\intlr{a}{b}\sin nx d\left( g(x)\right)} \leqslant 
					\frac{1}{n} \intlr{a}{b} \abs{d(g(x))} dx \xrightarrow[n \rightarrow +\infty]{} 0$ 
			\end{enumerate}
	\end{enumerate}

	Теперь докажем лемму для $f$. $\abs{\intl{\mathbb{R}}f(x) \cos nx} \leqslant 
	\intl{\mathbb{R}}\abs{f - g}\abs{\cos nx} dx + \abs{\intl{\mathbb{R}} g(x)\cos nx dx}$.
	\begin{enumerate}
		\item $\intl{\mathbb{R}}\abs{f - g}\abs{\cos nx} dx \leqslant
			\intl{\mathbb{R}}\abs{f - g} dx < \Epsilon $
		\item $\abs{\intl{\mathbb{R}} g(x)\cos nx dx} \xrightarrow[n \rightarrow +\infty]{} 0$
	\end{enumerate}
	Для синуса аналогично.
\end{proof}

Из этой леммы вытекает важное свойство рядов Фурье, описаное в следующем параграфе.

\subsection{Принцип локализации Римана}

\begin{theorem}[Принцип локализации Римана]
	Пусть $g, f$ - суммируемы и $2\pi$ - периодичны, а так же в $\delta$-окрестности некоторой точки $x$
	их значения совпадают, тогда 
	\begin{gather*}
		\lim\limits_{n\rightarrow +\infty} S_n(f, x) - S_n(g, x) = 0
	\end{gather*}
\end{theorem}

\begin{proof}
	Для простоты, представим, что у нас $x = 0$. Тогда
	\begin{gather*}
		S_n(f, x) - S_n(g, x) = \intl{Q} \left(f(x + t) - g(x + t) \right) D_n(t) dt = \\
		\intlr{-\pi}{-\delta} \left(f(x + t) - g(x + t) \right) D_n(t) dt +
		\intlr{\delta}{\pi} \left(f(x + t) - g(x + t) \right) D_n(t) dt
	\end{gather*}
	Будем оценивать такой интеграл(для остальных аналогично):
	\begin{gather*}
		\intlr{\delta}{\pi}f(x + t) \frac{1}{2\pi}
		\frac{\sin \left(n + \frac{1}{2} \right) t}{\sin \frac{t}{2}}
	\end{gather*}
	Перепишем $D_n(t)$:
	\begin{gather*}
		\frac{\sin \left(n + \frac{1}{2} \right) t}{\sin \frac{t}{2}} =
		\frac{\sin nt \cos \frac{t}{2} + \cos nt \sin \frac{t}{2}}{\sin\frac{t}{2}} = 
		\sin nt\ctg\frac{t}{2} + \cos nt
	\end{gather*}
	Далее, заметим важную вещь, которая заставляет все работать:
	При $\frac{\delta}{2} \leqslant \frac{t}{2} \leqslant \frac{\pi}{2}$, $\ctg\frac{t}{2}$ - ограничен.
	Получается, что $f(x + t)\ctg \frac{t}{2}$ и $f(x + t)$ - обе суммируемые функции, 
	а значит, для них выполнены условия леммы Римана-Лебега, отсюда получаем требуемое.
\end{proof}

\subsection{Теорема Фейера}

\begin{definition}[Сумма Фейера]
	\begin{gather*}
		\sigma_n(f, x) \defeq \frac{1}{n + 1} \sumlr{k = 0}{n} S_k(f, x)
	\end{gather*}
\end{definition}

Получим ее форму через интеграл:

\begin{gather*}
	\sigma_n(f, x) = \frac{1}{n + 1} \sumlr{k = 0}{n} \intl{Q} f(x + t)D_n(t) dt = 
	\intl{Q} f(x + t) \left( \frac{1}{n + 1} \sumlr{k = 0}{n} D_k(t)dt\right)
\end{gather*}

\begin{definition}[Ядро Фейера]
	\begin{gather*}
		\Phi_n(t) \defeq  \frac{1}{n + 1} \sumlr{k = 0}{n} D_k(t)dt
	\end{gather*}
\end{definition}

\begin{definition}[Интеграл Фейера]
	\begin{gather*}
		\sigma_n(f, x) = \intl{Q} f(x + t)\Phi_n(t)dt
	\end{gather*}
\end{definition}

\begin{statement}
	\begin{gather*}
		\Phi_n(t) = 
		\frac{1}{2\pi(n + 1)}\left(\frac{\sin(\frac{n + 1}{2} t)}{\sin\frac{t}{2}} \right)^2
	\end{gather*}
\end{statement}

\begin{theorem}[Теорема Фейера]
	Пусть $f \in L_1$, $f$  - периодична с периодом $T$. 
	Пусть $\\ \phi_x(t) = f(x + t) + f(x - t) - 2S$.
	Пусть $\frac{1}{t} \intlr{0}{t}\abs{\phi_x(t)} \xrightarrow[t \rightarrow 0]{} 0$.
	Тогда 
	\begin{gather*}
		\sigma_n(f, x) \xrightarrow[n \rightarrow 0]{} S
	\end{gather*}
\end{theorem}

\begin{proof}
	Пусть $h_n = \frac{1}{n + 1}$. $\sigma_n(f, x) - S 
	= \intlr{0}{\pi} \phi_x(t)\Phi(t)dt$.
	Разобьем интеграл на два $\intlr{0}{\pi} = \intlr{0}{h_n} + \intlr{h_n}{\pi}$.
	\begin{gather*}
		\abs{\sigma_n(f, x) - s} \leqslant \intlr{0}{h_n} \abs{\phi_x(t)} \Phi(t)dt +
		\intlr{h_n}{\pi} \abs{\phi_x(t)} \Phi(t)dt
	\end{gather*}
	Заметим, что $\abs{\sin nt} \leqslant n \abs{\sin t} \Rightarrow \Phi_n(t) \leqslant 
	\frac{1}{2\pi h_n}$, следовательно $\intlr{0}{h_n} \abs{\phi_x(t)} \Phi(t)dt \leqslant
	\frac{1}{2\pi} \frac{1}{h_n} \intlr{0}{h_n} \abs{\phi_x(t)}dt 
	\xrightarrow[n \rightarrow 0]{} 0\\$  
	to be continued...
\end{proof}


