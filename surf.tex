\section{Криволинейные интегралы}

\subsection{Определение}

Интегралы, которые будут рассмотрены в данном параграфе будут частными случаями интегралов по многообразиям от дифференциальных форм.

Итак, рассмотрим кривую $\Gamma : \langle a, b \rangle \rightarrow \mathbb{R}^3$. 
Здесь и далее считаем координатные функции непрерывно-дифференцируемыми, а дугу - спрямляемой 
(на всякий случай: эти понятия эквивалентны).
Напомним, что спрямляемость означает существование интеграла $l(\Gamma) = \intlr{a}{b} \sqrt{x'^2 + y'^2 + z'^2}dt$

Мы будет рассматривать 2 случая:
\begin{enumerate}
	\item
		$f: \Gamma \rightarrow \mathbb{R}\\$ 
		В силу спрямляемости дуги $\exists l(\wideparen{P_k P_{k + 1}})$.
		Как всегда, рассматриваем разбиение $\tau: a = t_0 < t_1 < \dots < t_n = b$. 
		У нас появилось множество точек $P_k = (x(t_k), y(t_k), z(t_k))$.
		Пусть $\widetilde{P}_k = (x(\widetilde{t}_k), y(\widetilde{t}_k), z(\widetilde{t}_k))$, 
		где $\widetilde{t}_k \in [t_k, t_{k + 1}]$.
		Составляем интегральную сумму:
		\begin{gather}
			\sigma(\tau) = \sumlr{k = 0}{n - 1} f(\widetilde{P}_k)l(\wideparen{P_k P_{k + 1}})
		\end{gather}
		\begin{definition}
			$rang\tau = \max\limits_{k} l(\wideparen{P_k P_{k + 1}})$
		\end{definition}
		\begin{definition}[Криволинейный интеграл первого рода]
			Eсли $\exists \lim\limits_{rang\tau \rightarrow 0} \sigma(\tau)$, и он не зависит от 
			выбора промежуточных разбиений, то он называется криволинейным интегралом первого рода
		\end{definition}
	\item
		$f: \Gamma \rightarrow \mathbb{R}^3$.
		$\tau$ и $\widetilde{P}_k$ определяем так же.
		Пусть
		\begin{gather*}
		\Delta x_k = x(t_{k + 1}) - x(t_k),\:
		\Delta y_k = y(t_{k + 1}) - y(t_k),\:
		\Delta z_k = z(t_{k + 1}) - z(t_k).
		\end{gather*}
		Составим три интегральные суммы:
		\begin{gather}
			\sigma_x(\tau) = \sumlr{k = 0}{n - 1}f_x(\widetilde{P}_k)\Delta x_k \\
			\sigma_y(\tau) = \sumlr{k = 0}{n - 1}f_y(\widetilde{P}_k)\Delta y_k \\
			\sigma_z(\tau) = \sumlr{k = 0}{n - 1}f_z(\widetilde{P}_k)\Delta z_k
		\end{gather}
		\begin{definition}[Криволинейный интеграл второго рода]
			Если $\exists \lim\limits_{\text{rang}\tau \rightarrow 0}\sigma_x(\tau) = I_x, 
			\exists \lim\limits_{\text{rang}\tau \rightarrow 0}\sigma_y(\tau) = I_y,
			\exists \lim\limits_{\text{rang}\tau \rightarrow 0}\sigma_z(\tau) = I_z$
			Причем, они не зависят от выбора промежуточных разбиений, 
			то они называются криволинейными интегралами второго рода по координатным функциям
			и обозначаются:
			\begin{gather*}
				\intl{\Gamma}f_x(x, y, z)dx \defeq I_x \\
				\intl{\Gamma}f_y(x, y, z)dy \defeq I_y \\
				\intl{\Gamma}f_z(x, y, z)dz \defeq I_z 
			\end{gather*}
		\end{definition}
\end{enumerate}

\subsection{Вычисление криволинейных интегралов первого рода}

\begin{theorem}[О вычислении криволинейных интегралов первого рода]
	Если $f$ - непрерывна вдоль $\Gamma$, и $\Gamma$ - гладкая, 
	то криволинейный интеграл первого рода существует, и равен
	\begin{gather*}
		\intl{\Gamma}f dl = \intlr{a}{b} f(x(t), y(t), z(t))\sqrt{x'^2 + y'^2 + z'^2} dt
	\end{gather*}
	
	\begin{proof}
		Мы составляли интегральные суммы вида:
		\begin{gather*}
			\sigma(\tau) = \sumlr{k = 0}{n - 1} f(\widetilde{P}_k)l(\wideparen{P_k P_{k + 1}})
		\end{gather*}
		Так как $f(\widetilde{P}_k) = f(x(\widetilde{t}_k), y(\widetilde{t}_k), z(\widetilde{t}_k))$, 
		а $l(\wideparen{P_k P_{k + 1}}) = \intlr{t_k}{t_{k + 1}} \sqrt{x'^2 + y'^2 + z'^2}dt 
		\Rightarrow \frac{dl}{dt} = \sqrt{x'^2 + y'^2 + z'^2}$, причем последняя производная 
		одна и та же для всех $k$. Тогда
		$\intl{\Gamma} fdl = \intlr{t_0}{t_1} f(x(t), y(t), z(t))l' dt =\\
		\intlr{t_0}{t_1} f(x(t), y(t), z(t))\sqrt{x'^2 + y'^2 + z'^2} dt $
	\end{proof}
\end{theorem}

\subsection{Вычисление криволинейных интегралов второго рода}

\begin{theorem}[О вычислении криволинейных интегралов второго рода]
	Если $f$ - непрерывна вдоль $\Gamma$, и $\Gamma$ - гладкая, то 
	\begin{gather*}
		\intl{\Gamma} f_x dx + f_ydy + f_zdz = 
		\intlr{a}{b} f_x(x(t), y(t), z(t))x' dt + 
					f_y(x(t), y(t), z(t))y' dt +
					f_z(x(t), y(t), z(t))z' dt
	\end{gather*}
\end{theorem}

\begin{proof}
	для простоты докажем формулу
	$\intl{\Gamma} f_xdx = \intlr{a}{b}f_x(x(t), y(t), z(t))x' dt$. 
	Исходная получается аналогичным доказательством двух оставшихся, и применением свойства линейности. 
	Рассмотрим следующую интегральную сумму:
	\begin{gather*}
		\sigma(\tau) = \sumlr{k = 0}{n - 1} f(\overline{P}_k)x'(\overline{t}_k)\Delta t_k
		\xrightarrow[\text{rang}\tau \rightarrow 0]{} \intlr{a}{b} f x' dt
	\end{gather*}
	Теперь оценим модуль разности: 
	\begin{gather*}
		\abs{\sigma_x(\tau) - \sigma(\tau)} \leqslant 
		\sumlr{k = 0}{n - 1} \abs{f_x(\widetilde{P}_k) - f_x(\overline{P}_k)}
		\abs{x'(\overline{t}_k)} \Delta t_k
	\end{gather*}
	По теореме Кантора, $f_x(t)$ - равномерно непрерывна на $[a, b]$, это значит, что
	\begin{gather*}
		\feps > 0 \exists \delta : \text{rang}\tau \leqslant \delta \Rightarrow 
		\forall t', t'' \abs{f_x(t') - f_x(t'')} \leqslant \Epsilon  \: \Rightarrow\\
		\Rightarrow \abs{\sigma_x(\tau) - \sigma(\tau)} \leqslant 
		\Epsilon \sumlr{k = 0}{n - 1} x'(\overline{t}_k) \Delta t_k \leqslant \Epsilon M
	\end{gather*}
\end{proof}
\subsection{Формула Грина}

Существует связь между криволинейным интегралом второго рода по замкнутому контуру и 
двойным интегралом по внутренности этого контура. Она выражается в следующей теореме:

\begin{theorem}[Формула Грина]
	Пусть $P$ и $Q$ - непрерывно-дифференцируемы в односвязной области $G$. Пусть $\Gamma = \partial G$.
	Тогда:
	\begin{gather*}
		\intl{\Gamma_{+}} Pdx + Qdy = 
		\iintl{G} \left( \frac{\partial Q}{\partial x} - \frac{\partial P}{\partial y} \right) dxdy
	\end{gather*}
\end{theorem}

$\Gamma_{+}$ - означает, что обход такой, что внутренность $G$ всегда слева.

\begin{lemma}
	Пусть $G = \left\{(x, y) : a\leqslant x \leqslant b, f(x) \leqslant y \leqslant g(x)  \right\} $
	Пусть в $G \: \exists P$ - непрерывная и $\exists \frac{\partial P}{\partial y}$ - непрерывная.
	Тогда:
	\begin{gather*}
		\intl{\partial G_{+}} Pdx = -\iintl{G} \frac{\partial P}{\partial y} dxdy 
	\end{gather*}
\end{lemma}

\begin{proof}
	По теореме Фубини:
	\begin{gather*} 
		-\iintl{G} \frac{\partial P}{\partial y} dxdy = 
		-\intlr{a}{b}dx\intlr{f(x)}{g(x)} \frac{\partial P}{\partial y} dy = \\
		-\intlr{a}{b}\left( P(x, g(x)) - P(x, f(x)) \right) dx
	\end{gather*}
	\begin{figure}[h]
	\centering
	\begin{tikzpicture}
		\draw[->] (-0.1,0) -- (4.2,0) node[right] {$$};
		\draw[->] (0,-0.1) -- (0,4.2) node[above] {$$};
		\draw[scale=0.5,domain=1:7,smooth,variable=\x] node at (1.55,4) {\rom{4}}plot ({1},{\x});
		\draw[scale=0.5,domain=1:7,smooth,variable=\x] node at (7.5, 4) {\rom{2}} plot ({8},{\x});
		\draw (0.5,0.5) .. controls (1.66,1.5) and (2.83,-0.5) .. (4,0.5);
		\node at (2.25, 0.8) {\rom{1}};
		\node at (3.4,  3.7) {$g(x)$};
		\draw (0.5,3.5) .. controls (1.66,4.5) and (2.83,3) .. (4,3.5);
		\node at (2.25, 3.4) {\rom{3}};
		\node at (1.15, 0.5) {$f(x)$};
	\end{tikzpicture}
	\caption{множество $G$}
	\end{figure}
	Разделим интеграл на 4:
	\begin{gather*}
		\intl{\partial G_{+}} P dx = \intl{\text{\rom{1}}}Pdx + 
		\intl{\text{\rom{2}}}Pdx + \intl{\text{\rom{3}}}Pdx + \intl{\text{\rom{4}}}Pdx
	\end{gather*}
	\begin{gather*}
		\text{\rom{1}}: \begin{cases}
							x = t \\ 
							y = f(t)
						\end{cases} 
		\Rightarrow
		\intl{\text{\rom{1}}} P dx = \intlr{a}{b} P(t, f(t)) dt \\
		\text{\rom{3}}: \begin{cases}
							x = t \\ 
							y = g(t)
						\end{cases} 
		\Rightarrow
		\intl{\text{\rom{3}}_{-}} P dx = \intlr{a}{b} P(t, g(t)) dt 
	\end{gather*}
	На \rom{2} и \rom{4} $x = \text{const} \Rightarrow \intl{\text{\rom{2}}} Pdx = 
	\intl{\text{\rom{4}}}Pdx = 0$.
	Складывая, получаем, что правая часть формулы из условия, равна левой части
\end{proof}


\begin{nb}
	Если $G = \left\{ c \leqslant y \leqslant d, f(y) \leqslant x \leqslant g(y) \right\}$, то 
	\begin{gather*}
		\intl{\partial G_{+}} Qdy = \iintl{G} \frac{\partial Q}{\partial x} dxdy
	\end{gather*}
\end{nb}
\begin{proof}[Доказательство формулы Грина]
	Возьмем две точки на границе $G$. Соединим их жордановой кривой, 
	через внутренность $G$, тогда $G$ разделится на две области - $G_1$ и $G_2$, и
	\begin{gather*}
		\intl{\Gamma_{+}}Pdx + Qdy = \intl{\partial G_{1+}} Pdx + Qdy + 
		\intl{\partial G_{2+}} P dx + Qdy
	\end{gather*}
	Если мы научимся доказывать теорему для каких-то конкретных разделений $G$ на $G_1$ и
	$G_2$, то, теорема юудет доказана.
	Можно показать, что $G$ можно разбить на области удовлетворяющие лемме. 
	По аддитивности и линейности интеграла, в сумме они дают, исходную формулу.
\end{proof}
\newpage

\section{Поверхностные интегралы}

В этом параграфе рассматриваем двумерные поверхноси в трехмерном пространстве:
\begin{gather*}
	S:\begin{cases}
		x = x(u, v) \\
		y = y(u, v) , & (u, v) \in G \subset \mathbb{R}^2 \\
		z = z(u, v)
	  \end{cases}
\end{gather*}
Можно говорить о двусторонних и односторонних поверхностях. 
Для этого сначала введем понятние нормали к поверхности:
\begin{definition}[Нормаль к поверхности]
	Пусть 
	\large
	\begin{gather*}
		k_u = \left( \pdiff{x}{u}, \pdiff{y}{u}, \pdiff{z}{u}\right),
		k_v = \left( \pdiff{x}{v}, \pdiff{y}{v}, \pdiff{z}{v}\right) \\
		\text{Тогда вектор нормали } n = k_u \times k_v = 
		\begin{vmatrix}
			i & j & k \\ 
			\pdiff{x}{u} & \pdiff{y}{u} & \pdiff{z}{u} \\
			\pdiff{x}{v} & \pdiff{y}{v} & \pdiff{z}{v} \notag
		\end{vmatrix}
	\end{gather*}
\end{definition}

\begin{definition}[Двусторонняя поверхность]
	Поверхность называется двусторонней если ее вектор нормали непрерывен на ней.
\end{definition}

По определению, модуль векторного произведения, равен площади параллелограмма, построенного на множителях, это значит, что можно считать площадь поверхностей следующим образом:
\begin{gather*}
	\overline{N}_x = \frac{D(y, z)}{D(u, v)} = 
	\begin{vmatrix}
		\pdiff{y}{u} & \pdiff{y}{v} \\
		\pdiff{z}{u} & \pdiff{z}{v} \notag
	\end{vmatrix}; 
	\overline{N}_y = \frac{D(z, x)}{D(u, v)} = 
	\begin{vmatrix}
		\pdiff{z}{u} & \pdiff{z}{v} \\
		\pdiff{x}{u} & \pdiff{x}{v} \notag
	\end{vmatrix}; 
	\overline{N}_z = \frac{D(x, y)}{D(u, v)} = 
	\begin{vmatrix}
		\pdiff{x}{u} & \pdiff{x}{v} \\
		\pdiff{y}{u} & \pdiff{y}{v} \notag
	\end{vmatrix} \\
	\text{Тогда площадь бесконечно малого сегмента поверхности равна}
	\sqrt{\left( \frac{D(y,z)}{D(u,v)} \right)^2 + 
	\left( \frac{D(z, x)}{D(u,v)} \right)^2 +
	\left( \frac{D(x, y)}{D(u,v)}\right)^2} dudv
\end{gather*}
Тогда площадь поверхности
\begin{gather*}
	mes(S) \defeq\iintl{G} 
	\sqrt{\left( \frac{D(y,z)}{D(u,v)} \right)^2 + 
	\left( \frac{D(z, x)}{D(u,v)} \right)^2 +
	\left( \frac{D(x, y)}{D(u,v)}\right)^2} dudv
\end{gather*}
Дальше как всегда, делаем разбиение:
$\\ \tau = t_{11} < t_{12} < \dots < t_{1n}\\ t_{21} < t_{22} < \dots < t_{2n}\\ 
\dots\\ t_{n1} < t_{n2} < \dots < t_{nn}, t_{ij} \in G, 
t_{1j}, t_{i1}, t_{nj}, t_{in}\in \partial G$ и 
$\\ S_{ij} = 
\left\{ (x, y, z) : x = x(u,v), y = y(u,v), z = z(u,v), (u,v) \in G, 
t_{ij} \leqslant u \leqslant t_{i + 1 j}, t_{ij} \leqslant v \leqslant t_{i j + 1}\right\},
\\ P_{ij} \in S_{ij}$.
Теперь у нас все готово для определения поверхностного интеграла.

\subsection{Поверхностный интеграл первого рода}

Рассмотрим функционал $f(x, y, z), (x, y, z) \in S$.
Можно составить интегральную сумму 
\begin{gather*}
	\sigma(\tau) = \suml{ij} f(P_{ij}) mes(S_{ij})
	\xrightarrow[\text{rang}\tau \rightarrow 0]{} \iintl{S}f(x, y, z)dS
\end{gather*}
И тогда
\begin{gather*}
	\iintl{S}f(x,y,z)dS = \iintl{G}f(x(u,v), y(u,v), z(u, v))
	\sqrt{\left( \frac{D(y,z)}{D(u,v)} \right)^2 + 
	\left( \frac{D(z, x)}{D(u,v)} \right)^2 +
	\left( \frac{D(x, y)}{D(u,v)}\right)^2} dudv
\end{gather*}

\subsection{Поверхностный интеграл второго рода}

Рассмотрим функцию $\vec{f} : \mathbb{R}^3 \rightarrow \mathbb{R}^3$. Интегральные суммы определяются аналогично - умножаем координатную функцию на элемент площади. Тогда
\begin{gather*}
	\iintl{S} \vec{f}_xdydz + \vec{f}_ydzdx + \vec{f}_z dxdy \defeq 
	\iintl{S}\left( \vec{f}, \vec{n}\right)dS
\end{gather*}

Приведем некоторые определения из дифференциального исчисления:

\begin{definition}[Оператор Гамильтона (набла)]
	\begin{gather*}
		\nabla \defeq 
		\left( \pdiff{}{x}, \pdiff{}{y}, \pdiff{}{z}\right)
	\end{gather*}
\end{definition}

\begin{definition}[Градиент скалярного поля]
	\begin{gather*}
		\nabla f = \left( \pdiff{f}{x}, \pdiff{f}{y}, \pdiff{f}{z}\right)
		\defeq \grad f
	\end{gather*}
\end{definition}

\begin{definition}[Дивергенция векторного поля]
	\begin{gather*}
		\left( \nabla, \vec{F}\right) = 
		\pdiff{\vec{F}_x}{x} + \pdiff{\vec{F}_y}{y} + \pdiff{\vec{F}_z}{z} \defeq \Div \vec{F}
	\end{gather*}
\end{definition}

\begin{definition}[Ротор векторного поля]
	\begin{gather*}
		\nabla \times \vec{F} = 
		\begin{vmatrix}
			i & j & k \\
			\pdiff{}{x} & \pdiff{}{y} & \pdiff{}{z} \\
			\vec{F}_x & \vec{F}_y & \vec{F}_z \notag
		\end{vmatrix}
		\defeq \rot \vec{F}
	\end{gather*}
\end{definition}

\begin{definition}[Оператор Лапласа (Лапласиан)]
	\begin{gather*}
	\Delta \defeq 
		 \frac{\partial^2}{\partial x^2} +
		 \frac{\partial^2}{\partial y^2} + 
		 \frac{\partial^2}{\partial z^2}
	\end{gather*}
\end{definition}

\begin{theorem}[О треугольниках]
	\begin{gather*}
		\nabla^2 = \Delta
	\end{gather*}
\end{theorem}

\begin{proof}
	\begin{gather*}
		\nabla^2 = \left( \nabla , \nabla \right) = 
		\frac{\partial^2}{\partial x^2} +
		\frac{\partial^2}{\partial y^2} + 
		\frac{\partial^2}{\partial z^2}
		= \Delta
	\end{gather*}
\end{proof}

Следующая теорема позволяют сводить контурные интегралы второго рода к 
поверхностным интегралам второго рода.

\begin{theorem}[Теорема Стокса (без доказательства)]
	\begin{gather*}
		\oint\limits_{\partial S_{+}} \vec{F}_x dx + \vec{F}_y dy + \vec{F}_z dz =
		\iintl{S} \left( \rot \vec{F}, \vec{n}\right)dS
	\end{gather*}
\end{theorem}
Теорема Остроградского-Гаусса позволяет вычислять криволинейные интегралы 
по замкнутому контуру через поверхностные интегралы второго рода
\begin{theorem}[Теорема Остроградского-Гаусса (без доказательства)]
	\begin{gather*}
		\iintl{\partial S} \left( \vec{F}, \vec{r}\right)dS =
		\iiint\limits_{S} \Div \vec{F} dxdydz
	\end{gather*}
\end{theorem}

\begin{nb}
	Все эти теоремы являются частным случаем одной, доказаной Анри Картаном в 
	\rom{20} веке:
	\begin{gather*}
		\intl{\partial S} \omega = \intl{S} d\omega
	\end{gather*}
	Где $S$ - многообразие, $\omega$ - дифференицальная форма
\end{nb}
